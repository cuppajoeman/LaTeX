\documentclass[11pt]{book}

\RequirePackage{silence}
\WarningFilter{remreset}{The remreset package}

\title{MAT246 - Concepts in Abstract Mathematics}
\author{Callum Cassidy-Nolan}

% Packages
\usepackage{amsmath}
\usepackage{amssymb}
\usepackage{mathtools}
\usepackage{xcolor}
\usepackage{amsthm}
\usepackage{thmtools}
\usepackage{amsfonts}
\usepackage{geometry}
\usepackage{gauss}
\usepackage{pifont}
\usepackage{hyperref}
\usepackage{witharrows}
\usepackage{cleveref}
\usepackage{tikz}
\usepackage{bm}
\usepackage{todonotes}
\usepackage{enumitem}
\usepackage{mdframed}
\usetikzlibrary{patterns,angles,quotes,decorations.pathreplacing}
\hypersetup{
    colorlinks=true, %set true if you want colored links
    linktoc=all,     %set to all if you want both sections and subsections linked
    linkcolor=blue,  %choose some color if you want links to stand out
}


% BlackSquare for proofs
\renewcommand{\qedsymbol}{$\blacksquare$}

\theoremstyle{definition}

% Matricies
\newcommand\mat[2][b]{\begin{#1matrix}#2\end{#1matrix}}

% Augmented matrix
\makeatletter
\renewcommand*\env@matrix[1][*\c@MaxMatrixCols c]{%
  \hskip -\arraycolsep
  \let\@ifnextchar\new@ifnextchar
  \array{#1}}
\makeatother

% Modular Arithmetic
\newcommand{\Mod}[1]{\ (\mathrm{mod}\ #1)}

% Automatic Parenthesis scaling
\delimitershortfall-1sp
\usepackage{mleftright}
\mleftright % make \left & \right behave like \mleft & \mright

% Theorems
\newtheoremstyle{break}
  {\topsep}{\topsep}%
  {\itshape}{}%
  {\bfseries}{}%
  {\newline}{}%

\theoremstyle{break}

\newtheorem{remark}{Remark}[section]

\newtheorem{ver}{Verfication}[section]

\newtheorem{ex}{Exercise}[section]

\newtheorem{eg}{Example}[section]

% definition env
\newmdtheoremenv{defn}{Definition}

% Note env
\newmdtheoremenv{nt}{Note}

% definition env no num
\newtheorem*{defnnonum}{Definition}

% theorem envs
\newtheorem{thm}{Theorem}

% theorem envs without counter

\newtheorem{propo}[thm]{Proposition}

\newtheorem{crly}[thm]{Corollary}

\newtheorem{lemma}[thm]{Lemma}

\newtheorem{axiom}[thm]{Axiom}

\newtheorem*{thmnonum}{Theorem}

\newtheorem*{propononum}{Proposition}

\newtheorem*{crlynonum}{Corollary}

\newtheorem*{lemmanonum}{Lemma}

\newtheorem*{axiomnonum}{Axiom}


\newtheorem{note}{Note}[section]

\newtheorem{mnote}[note]{Note}

\newtheorem*{notation}{Notation}
    % warning env
\newtheorem*{warning}{Warning}

% So todo's don't get cut off
\setlength{\marginparwidth}{3cm}

% Define cuboids
\tikzset{
  annotated cuboid/.pic={
    \tikzset{%
      every edge quotes/.append style={midway, auto},
      /cuboid/.cd,
      #1
    }
    \draw [every edge/.append style={pic actions, densely dashed, opacity=.5}, pic actions]
    (0,0,0) coordinate (o) -- ++(-\cubescale*\cubex,0,0) coordinate (a) -- ++(0,-\cubescale*\cubey,0) coordinate (b) edge coordinate [pos=1] (g) ++(0,0,-\cubescale*\cubez)  -- ++(\cubescale*\cubex,0,0) coordinate (c) -- cycle
    (o) -- ++(0,0,-\cubescale*\cubez) coordinate (d) -- ++(0,-\cubescale*\cubey,0) coordinate (e) edge (g) -- (c) -- cycle
    (o) -- (a) -- ++(0,0,-\cubescale*\cubez) coordinate (f) edge (g) -- (d) -- cycle;
    \path [every edge/.append style={pic actions, |-|}]
    (b) +(0,-5pt) coordinate (b1) edge ["\cubex \cubeunits"'] (b1 -| c)
    (b) +(-5pt,0) coordinate (b2) edge ["\cubey \cubeunits"] (b2 |- a)
    (c) +(3.5pt,-3.5pt) coordinate (c2) edge ["\cubez \cubeunits"'] ([xshift=3.5pt,yshift=-3.5pt]e)
    ;
  },
  /cuboid/.search also={/tikz},
  /cuboid/.cd,
  width/.store in=\cubex,
  height/.store in=\cubey,
  depth/.store in=\cubez,
  units/.store in=\cubeunits,
  scale/.store in=\cubescale,
  width=10,
  height=10,
  depth=10,
  units=cm,
  scale=.1,
}

% highlighting shortcuts
\newcommand{\hlimpo}[1]{\textcolor{red}{\textbf{#1}}}
\newcommand{\hlwarn}[1]{\textcolor{yellow}{\textbf{#1}}}
\newcommand{\hldefn}[1]{\textcolor{blue}{\index{#1}\textbf{#1}}}
\newcommand{\hlnotea}[1]{\textcolor{green}{\textbf{#1}}}
\newcommand{\hlnoteb}[1]{\textcolor{cyan}{\textbf{#1}}}
\newcommand{\hlb}[2]{\colorbox{#1!30!background}{#2}}
\newcommand{\hlbnotea}[1]{\hlb{green}{#1}}
\newcommand{\hlbnoteb}[1]{\hlb{cyan}{#1}}
\newcommand{\hlbnotec}[1]{\hlb{yellow}{#1}}
\newcommand{\hlbnoted}[1]{\hlb{magenta}{#1}}
\newcommand{\hlbnotee}[1]{\hlb{red}{#1}}
\newcommand{\WTP}{\textcolor{bwhite}{WTP} }
\newcommand{\WTS}{\textcolor{bwhite}{WTS} }



\begin{document}

\maketitle

\tableofcontents

\renewcommand{\listtheoremname}{List of Definitions}
\listoftheorems[ignoreall,show={defn}]


\renewcommand{\listtheoremname}{\textsl{List of Theorems}}
\listoftheorems[ignoreall,
show={axiom,lemma,thm,crly,propo}
]


\chapter{Lecture 1}%
\label{chp:lecture_1}
% chapter lecture_1

\section{Induction}%
\label{sec:induction}
% section induction

\begin{nt}
    \[
    \mathbb{N} = \left\{ 1, 2, 3, \ldots  \right\} 
    \]
\end{nt}

\begin{defn}[The principle of mathmatical induction ]\index{The principle of mathmatical induction }\label{defn:the_principle_of_mathmatical_induction_}
suppose $S \subseteq \mathbb{N} $  

If
\begin{itemize}
    \item $1\in S$ 
    \item $k + 1 \in S$ whenever $k \in S$ 
\end{itemize}
Then
\[
    \boxed{ S = \mathbb{N} }
\]

\end{defn}

The principle of mathmatical induction is simply saying if $1$ is in $S$ then $2, 3, \ldots $ is also in S

\begin{eg}
    Prove 
    \[
        \forall n \in \mathbb{N} , \underbracket{1^2  + 2^2  + \ldots  + n^2 = \frac{n\left( n + 1 \right) \left( 2n + 1 \right) }{6}}_{\chi} 
    \]
    \begin{proof}
    $ $\newline
        Let $S= \left\{ n \in \mathbb{N} : \chi \text{ holds }  \right\} $ At this point we don't know what $S$ consists of but we must show it is $\mathbb{N} $, then we can conclude that the formula holds for all natural numbers. We commence by verifying that $1\in S$, we have 
        \[
            1^2 = \frac{1\left( 1 + 1 \right) \left( 2 + 1 \right) }{6}
        \]
        both the right hand side and left hand side are equal to eachother, so the formula holds for $1$. 

        We will now show if $k \in S$ then $k + 1 \in S$. We assume that $k \in S$, that is :
        \[
            1^2  + 2^2  + \ldots  + k^2 = \frac{k\left( k + 1 \right) \left( 2k + 1 \right) }{6}
        \]
        We observe that if we add $k + 1$ to both sides of the above equation we get the left hand side, of what we want to prove.
        \begin{align*}
            1^2  + 2^2  + \ldots  + k^2  + \left( k + 1 \right) ^2 &= \frac{k\left( k + 1 \right) \left( 2k + 1 \right) }{6} + \left( k + 1 \right) ^2 \\
            &= \frac{k\left( k + 1 \right) \left( 2k + 1 \right)  + 6\left( k + 1 \right) ^2 }{6}  \\ 
            &= \frac{\left( k + 1 \right) \left( k\left( 2k + 1 \right)  + 6\left( k + 1 \right)  \right) }{6}  \\ 
            &= \frac{\left( k + 1 \right) \left( 2k^2  + 7k + 6 \right) }{6}  \\ 
            &= \frac{\left( k + 1 \right) \left( k + 2 \right) \left( 2k + 3 \right) }{6} \\ 
            &= \frac{\left( k + 1 \right) \left( \left( k + 1 \right)  + 1  \right) \left( 2\left( k + 1 \right)  + 1 \right) }{6}  \\ 
        \end{align*}
        After working out the right hand side it is the original formula with $k + 1$ subbed in. Therefore we have shown that  if $k \in S$ then $k + 1 \in S$ as wanted, thus by the principle of mathmatical induction 
        \[
        S= \mathbb{N} 
        \]
        .
    \end{proof}
\end{eg}

\newpage

\begin{defn}[Extended principle of mathmatical induction ]\index{Extended principle of mathmatical induction }\label{defn:extended_principle_of_mathmatical_induction_}
    This is the same as normal induction, though now we don't have to start with 1.
    If  
    \begin{itemize}
        \item Let $n_{0} \in \mathbb{N} , n_{0} \in S$
        \item $k \in S \implies k + 1 \in S$ 
    \end{itemize}
    Then
    \[
    \boxed{S \supseteq \left\{ n_{0} , n_{0}  + 1, \ldots  \right\}  }
    \]
    Observe that $S$ is only a subset of these numbers as these are the ones that are guarenteed to be in $S$, there may be others. 
\end{defn}

\begin{eg}
    Prove for all integers $n$ greater than or equal to 7 that the following holds:
    \[
        \underbracket{ n! \ge 3^{n} }{\chi}
    \]
    \begin{proof}
    $ $\newline
        Let $S$ be the set of all natural numbers that $\chi $ holds for. We verify that $7 \in S$ 
        \[
        \underbracket{7!}_{5040} \ge \underbracket{3^{7} }_{2187} 
        \]
        therefore 7 satisfies $\chi$ and so $7 \in S$.
        Let $k \in \mathbb{N} $,  we assume $\chi$ holds for $k$, that is 
        \[
        k! \ge 3^{k} 
        \]
        We will prove 
        \[
            \left( k + 1 \right) ! \ge 3^{k + 1} 
        \]
        We observe that $\left( k + 1 \right) ! = \left( k + 1 \right) k!$, but recall that we assumed that $k! \ge 3^{k} $ so we have 
        \begin{align*}
            k!\left( k + 1 \right) &\ge 3^{k} \left( k + 1 \right) \\
            \intertext{Recall that $k\ge 7$ }
            &\ge 3^{k} 8   \\ 
            &\ge 3^{k + 1}
        \end{align*}
        Therefore, we've shown that
        \[
            \left( k + 1 \right) ! \ge 3^{k + 1} 
        \]
        as required, and so 
        \[
        S \supseteq \left\{ 7, 8, 9, \ldots  \right\} 
        \]
    \end{proof}
    
\end{eg}

% section induction (end)

% chapter lecture_1 (end)

\chapter{Lecture 2}%
\label{chp:lecture_2}
% chapter lecture_2

\begin{defn}[Well Ordering Principle]\index{Well Ordering Principle}\label{defn:well_ordering_principle}
   Every subset of $\mathbb{N} $ other than $\emptyset $ has a smallest      element. 
\end{defn}

\section{Proof of Induction}%
\label{sec:proof_of_induction}
% section proof_of_induction

\begin{remark}
    We accepted the Principle of Mathematical Induction, though we should prove it.
\end{remark}

Recall, the Principle of Mathematical Induction, suppose $S \subseteq N$, if  
\begin{itemize}
    \item $1 \in S$ 
    \item $k + 1 \in S$ whnever $k\in S$ 
\end{itemize}
then 
\[
S= \mathbb{N} 
\]

We'll prove the statement
\begin{proof}
$ $\newline
    Let $T= \left\{ n\in \mathbb{N} : n \not\in S  \right\} $. suppose that $T\neq \emptyset $,  therefore by the Well Ordering Principle we know that $T$ has a smallest element, let $n_{0} $ be that element. Note that $n_{0} \in \mathbb{N}, n_{0} \neq 1 $ since $1\in S \therefore 1 \not\in T$, therefore $n_{0} \ge 2$.\\
    since $n_{0} \ge 2 $ we know $n_{0}  - 1 \in \mathbb{N} $ and that $n_{0}  - 1 \not\in T$ since $n_{0} $ is the smallest element in $T$. 
    \[
    n_{0}  - 1 \not\in T \implies n_{0}  - 1 \in S
    \]
    But by property 2, of $S$ we know that if $n_{0} \in S$ then $n_{0} \in S$,  though this is a contradiction as $n_{0} \not\in S$\\
    Therefore $T = \emptyset $ and $S = \mathbb{N} $ 
\end{proof}


% section proof_of_induction (end)

\section{Division}%
\label{sec:division}
% section division

\begin{defn}[Divides]\index{Divides}\label{defn:divides}
    for $a, b \in \mathbb{N} $ we say that $a$ divides $b$ if there exists a $c \in \mathbb{N} $ such that 
    \[
    b= ca
    \]
    And we say 
    \[
    a \mid b
    \]
\end{defn}
\begin{remark}
    $2 \cdot 3.5 = 7$, though our definition is only for natural numbers, since no $c \in \mathbb{N} $ gives $2 \cdot c = 7$ 
\end{remark}

\begin{defn}[Prime]\index{Prime}\label{defn:prime}
    $p \in \mathbb{N} $ is \underline{prime} if the only divisor of $p$ are $1 \text{ and } p$ and $p\neq 1$ 
\end{defn}

\paragraph{Example} 
\begin{itemize}
    \item 7 is prime, since the only divisor is 1 and 7
    \item 10 is not prime, 2 and 5 divide 10
\end{itemize}

\begin{thm}[Product of Primes]\index{Product of Primes}\label{thm:product_of_primes}
    for all $n \in \mathbb{N} ,n \neq 1$ $n$ can be written as a product of primes 
\end{thm}

\paragraph{Example} 
\begin{itemize}
    \item $42 = 2  \cdot 3 \cdot 7$ 
    \item $12 = 3  \cdot 2^{2} $ 
\end{itemize}

\newpage

\begin{defn}[Complete Induction]\index{Complete Induction}\label{defn:complete_induction}
    Let $S\subseteq \mathbb{N} $ 
    \begin{itemize}
        \item if $n_{0} \in S$ 
        \begin{itemize}
            \item and $k + 1 \in S$ when $n_{0} , n_{0}  + 1, \ldots , k \in S$ 
        \end{itemize}
        Then 
        \[
        S \supseteq \left\{ n_{0} , n_{0}  + 1, \ldots  \right\} 
        \]
    \end{itemize}
\end{defn}

We will prove the product of primes theorem
\begin{proof}
$ $\newline
    Let $S= \left\{ n \in \mathbb{N} : \text{ theorem holds for } n \right\} $ we will prove 
    \[
    S= \mathbb{N} 
    \]
    \begin{itemize}
        \item 2, is prime therefore it is a product of primes and so the Base Case holds. 
        \item We assume if $2, 3, \ldots , k \in S$ then $k + 1 \in S$ 
            \begin{itemize}
                \item \textbf{Case 1}: $k + 1$ is prime, then we are done like the base case
                \item \textbf{Case 2}: $k + 1$  is not prime, then there exists an $m \in \mathbb{N} $ such that $1 < m < k + 1$ and $m \mid k + 1$ by definition this means 
                    \[
                    k + 1 = c \cdot m, \text{ for some } c \in \mathbb{N} 
                    \]
                    observe that $1 < c < k + 1$ since if $c = 1$,$c= k + 1$ or if larger we get a contradiction.\\
                    Therefore we can use the Induction Hypothesis on $c \text{ and } m$ to write thpem both as a product of primes, multiplying them together gives us a new product of primes equal  to $k + 1$ as required. 
            \end{itemize}
    \end{itemize}
Therefore by the principle of complete induction we can say that 
\[
S \supseteq \left\{ 2, 3, \ldots  \right\} 
\]
though we want to show that $S= \left\{ 1, 2,3, \ldots  \right\} $ observe  that $1$ is not a product of primes as it is not prime and also not composite, therefore $1 \not\in S$ so $S= \left\{ 2, 3, \ldots  \right\} $  
\end{proof}

The intuition behind this proof comes from the fact that if we take a number say $24$ it is either prime  or not, in this case it is not, and we can write it as $24 = 6  \cdot 4$ then by an inductive argument, we already know that $6 $  and $4$ are already product of primes so we are done. We will show next that in fact this is a unique product.

% section division (end)

% chapter lecture_2 (end)

\chapter{Lecture 3}%
\label{chp:lecture_3}
% chapter lecture_3

Recall from last lecture we showed that every natural number besides 1 can be written as a product of primes. Thus we have the following 
\[
\forall n \in \mathbb{N} , n \ge 1 \implies n \text{ is divisible by some prime } 
\]
Let's call the above $\alpha $ 


\section{There is no largest prime}%
\label{sec:there_is_no_largest_prime}
% section there_is_no_largest_prim

\begin{proof}
$ $\newline
    suppose by contradiction $p$ is the largest prime, that is 
    \[
    \left\{ 2, 3, \ldots , p \right\} 
    \]
    are all the primes.
    Let $m= \left( 2 \cdot 3 \cdot \dotsm  \cdot p \right)  + 1$,  we note that for any $j \in \left\{ 2, 3, \ldots ,p \right\} $ they must not division $m$ as they each of a remainder of 1. We observe that $m \ge 1$ thus by $\alpha $ we know that there exists some $q \in \mathbb{N} $ where $q$ is prime such that 
    \[
    q \mid m
    \]
    So then $q \neq 2, 3, \ldots ,p$ and so we have found a new prime, which contradicted that we had found all primes, so we get a contradiction, therefore there is no largest prime.
\end{proof}

% section there_is_no_largest_prime (end)

% chapter lecture_3 (end)

\chapter{Lecture 4}%
\label{chp:lecture_4}
% chapter lecture_4

\begin{thm}[Fundamental Theorem of Arithmetic]\index{Fundamental Theorem of Arithmetic}\label{thm:fundamental_theorem_of_arithmetic}
    Every natural number other than 1, is a product of primes (proved last lecture) and the primes in the product are unique (including multiplicity)  except for the order in which they occur.
\end{thm}

Recall given $n\in \mathbb{N}, n \neq 1, n $ is a product of primes that is 
\[
n = p_1 p_2 \dotsb p_{k - 1} p_{k}
\]

for example 
\[
180 = 9 \cdot 10 \cdot 2= 3^{2} 5^{1} 2^{2} 
\]

So equivalently we have 
\[
\forall n \in \mathbb{N} , 1 < n \implies n = p_{1} ^{\alpha _{1} } p_{2} ^{\alpha _{2} } \dotsm p_{k} ^{\alpha _{k} }  
\]
where $p_{i} $ are distinct primes and $\alpha _{i} \in \mathbb{N} $ 

We will prove that the prime factorization of any natural number greater than 1 has is unique by contradiction.

\newpage

\begin{proof}
$ $\newline
We commence
\begin{itemize}
    \item Suppose there are some numbers with two distinct factorizations into primes.
    \item Let $\mathcal{X} $ be the set of these numbers, observe that $\mathcal{X} \subseteq \mathbb{N} $ thus by the Well Ordering Principle we let $n$ be the smallest such number in $\mathcal{X} $, we have 
        \[
        n= p_1 p_2 \dotsb p_{k - 1} p_{k} = q_1 q_2 \dotsb q_{l - 1} q_{l}
        \]
        (Note here we aren't using powers, but we allow for repeated primes, and that $p_{i} ,q_{j} $ are primes)
    \item Suppose that the two product of primes share at least one factor, say $p_{r} = q_{r} $, then in each product of primes they cancel out and we get that a new smaller number that can be written as a product of primes, though this would cause a contradiction since we assumed $n$ was the smallest such number with this property. 
        \begin{itemize}
            \item Therefore all the $p_{i} $ are different than the $q_{j} $ 
        \end{itemize}
    \item Since we know $p_{i} \neq q_{j} $ then specifically $p_{1} \neq q_{1} $ if this is true there are two cases either $p_{1} \le q_{1} $ or $p_{1} \ge q_{1} $.
    \item \textbf{Case 1}: $p_{1} < q_{1} $ 
        \begin{itemize}
            \item We note $n = q_1 q_2 \dotsb q_{l - 1} q_{l} > p_1 q_2 \dotsb q_{l - 1} q_{l}$ 
            \item $p_1 q_2 \dotsb q_{l - 1} q_{l} < n \Leftrightarrow 0 < n - p_1 q_2 \dotsb q_{l - 1} q_{l}$ 
            \item Note that $p_{i} , q_{j} \in \mathbb{N} $ so the product of any of them is also an element of the naturals, and then also $n - p_1 q_2 \dotsb q_{l - 1} q_{l} \in \mathbb{N}$.
        \end{itemize}
    \item Note that $m < n \implies m$ has a unique factorization into primes \todo{Why?}, we know
        \begin{itemize}
            \item $m = p_1 p_2 \dotsb p_{k - 1} q_{k}  - q_1 q_2 \dotsb q_{l - 1} q_{l}= p_{1} \left( p_2 \dotsb p_{k - 1} p_{k} - q_2 \dotsb q_{l - 1} q_{l} \right) $ 
            \item $m = q_1 q_2 \dotsb q_{l - 1} q_{l} - p_1 q_2 \dotsb q_{l - 1} p_{l} = \left(  q_2 \dotsb q_{l - 1} q_{l} \right) \left( q_{1}  - p_{1}  \right) $ 
            \begin{itemize}
                \item Together
                    \[
 p_{1} \left( p_2 \dotsb p_{k - 1} p_{k} - q_2 \dotsb q_{l - 1} q_{l} \right) = \underbracket{\left(  q_2 \dotsb q_{l - 1} q_{l} \right)}_{\chi}  \left( q_{1}  - p_{1}  \right) 
                    \]
            \end{itemize}
        \end{itemize}
    \item The left hand side of the above tells us that $p_{1} $ is a prime factor of $m$ which means it is also a factor of the right hand side.
        \begin{itemize}
            \item But observe $p_{1} \nmid \chi$ since $p_{1} \neq q_{j} $ 
            \item Therefore it must be that $p_{1} | \left( q_{1}  - p_{1}  \right) $ this is true if and only if $p_{1} \mid q_{1} $ since $q_{1} $ is prime this means that $p_{1} = 1$ or $p_{1} = q_{1} $ either of which are contradictions, therefore \todo{What does this contradict?} 
        \end{itemize}
\end{itemize}
\end{proof}

\paragraph{Techniques Used in this proof} 
\begin{itemize}
    \item Here
\end{itemize}

\begin{defn}[Composite]\index{Composite}\label{defn:composite}
    A natural number $c$ is called composite if
    \begin{align*}
        c \neq 1 && c \text{ is not prime } 
    \end{align*}
\end{defn}

Q: Can we find 20 consecutive composite numbers?

Yes, consider 
\[
21! + 2, 21!  + 3, \ldots , 21!  + 21
\]
Observe 2 divides the first number, 3, divides the next on until 21, giving us 20 composites

Q: Can we find $k$ consecutive composite numbers?

Yes, using hte same method we have 
\[
    \left( k + 1 \right)!  + 2, \left( k + 1 \right) !  + 3, \ldots , \left( k + 1 \right) !  + k + 1
\]
thus we conclude there are arbitrary long stretches of composite numbers.

\section{Modular Arithmetic}%
\label{sec:modular_arithmetic}
% section modular_arithmetic

For some intuition, consider military time, if somone tells us it's 15 o clock we know that this is equivalent to 3 o clock, and this will help us represent this type of situation

First we define the integers that is 
\[
\mathbb{Z} = \left\{ 0, \pm 1, \pm 2, \ldots  \right\} 
\]

\begin{defn}[Congruence]\index{Congruence}\label{defn:congruence}
    Let $a, b \in \mathbb{Z} $,  $m \in \mathbb{N} $.  if 
    \[
        m \mid \left( a - b \right) 
    \]
    then we say $a$ is \underline{congruent} to $b$ and we write
    \[
    a \equiv b \Mod{m}
    \]
\end{defn}

\paragraph{Example} 
\begin{itemize}
    \item Let $m = 12$ (the hours on the clock)
        \begin{itemize}
            \item it follows that $13 \equiv 1 \Mod{12} $ since $12 \mid \left( 13  - 1 \right) $ 
            \item $14 \equiv 2 \Mod{12} $ and $23 \equiv -1 \Mod{12} $ 
            \item So this shows us in the world of the clock some numbers are the ``same"
        \end{itemize}
    \item $m = 2$ 
        \begin{itemize}
            \item We observe 
                \begin{itemize}
                    \item $0 \equiv 0 \Mod{2} \Leftrightarrow 2 \mid \left( 0  - 0 \right) $ which is true since we take $c = 0$ in the definition of divisibility.
                    \item $1 \equiv 1 \Mod{2} $ 
                    \item $2 \equiv 0 \Mod{2} $ 
                    \item $3 \equiv 1 \Mod{2} $ 
                    \item $4 \equiv 0 \Mod{2} $ 
                    \item $5 \equiv 1 \Mod{2} $ 
                \end{itemize}
        \end{itemize}
\end{itemize}

% section modular_arithmetic (end)

% chapter lecture_4 (end)


\end{document}
