\documentclass[11pt]{article}

\usepackage{amsmath, amsfonts, amsthm}

\theoremstyle{definition}

\newtheorem{ver}{Verfication}[section]

\begin{document}
\section{Chapter 1}%
\label{sec:chapter_1}
% section chapter_1

\subsection{Complex Numbers}%
\label{sub:complex_numbers}
% subsection complex_numbers

\begin{ver}
    commutativity
    \begin{equation*}
        w + z = z + w \text{ and } wz = zw \text{ for all } w, z \in \mathbb{C}
    \end{equation*}
    We know that $w = \alpha + \beta i  \text{ and } z = j + k i  $ and so then 
    \begin{equation*}
        w + z = \alpha + \beta i + j + k i = \left( \alpha + j \right) + \left( \beta +k \right)i = \left( j + \alpha  \right) + \left( k + \beta  \right) = z + w
    \end{equation*}
\end{ver}

\begin{ver}
    associativity 
    \begin{equation*}
        \left( z_1 + z_2 \right) + z_3 = z_1 + \left( z_2 + z_3 \right) \text{ and } \left( z_1 z_2 \right)z_3 = z_1\left( z_2z_3 \right) \text{ for all } z_1, z_2, z_3 \in \mathbb{C}
    \end{equation*}
    We know $z_1 = \alpha + \beta i, z_2 = j + k i, z_3 = l + m i$ and thus 
    \begin{align*}
        \left( z_1 + z_2 \right) + z_3 &= \alpha  + j + \left( \beta  + k \right) i + l + m i \\
                                       &= \left( \alpha + j + l \right) + \left( \beta + k + m \right) i \\
                                       &= z_1 + \left( z_2 + z_3 \right) 
    \end{align*}

    Now we'll do multiplication, we know
    \begin{align*}
        \left( z_1z_2 \right)z_3 &= \left(   \alpha  \cdot j - \beta  \cdot k + \left( \alpha k + \beta j \right) i \right) \left( l + m i \right) \\
                                 &= ajl - blk - \left( \alpha km + \beta jm \right) + \left( \alpha kl + \beta jl + ajm \right) i
    \end{align*}
    Now we also have
    \begin{align*}
        z_1\left( z_2z_3 \right) &= \alpha + \beta i\left( jl - km + \left( kl + jm \right)i \right)\\
                                 &= ajl - akm - \left( \beta kl + \beta jm \right) + \left( \beta jl + akm + ajm \right) i
    \end{align*}
    And due to the associativity of $\mathbb{R}$ then we can say $\left( z_1z_2 \right)z_3 = z_1 \left( z_2z_3 \right)$ 
\end{ver}

\begin{ver}
    identities
    \begin{equation*}
        z + 0 = z \text{ and } z1 = z \text{ for all } z \in \mathbb{C}
    \end{equation*}
    We know that 
    \begin{align*}
        z + 0 &= \alpha + \beta i + 0 + 0 i\\
              &= \left( \alpha + 0 \right) + \left( \beta + 0 \right) i \\
              &= \alpha + \beta i\\
              &= z  
    \end{align*}
    For multiplication we have
    \begin{align*}
        z1 &=\alpha + \beta i\left( 1 + 0i \right)\\
           &= \alpha - \beta 0 + \left( \beta + 0\alpha  \right)\\
           &= \alpha + \beta i\\
    \end{align*}
\end{ver}

\begin{ver}
    additive inverse\\
    for every $z \in \mathbb{C}$ there is a unique $w \in \mathbb{C}$ such that $z + w = 0$ 

    Let $z \in \mathbb{C}$ and so $z = \alpha + \beta i$ now we'll take $w = -\alpha  + -\beta  i$ 
    \begin{align*}
        z + w &= \alpha + \beta i + -\alpha  + -\beta  i\\
              &= \left( \alpha -\alpha  \right) + \left( \beta - \beta  \right)i\\
              &=0 + 0 i\\
              &=0
    \end{align*}
    To show that our choice of $w$ was unique assume there is another solution namely $w = j + k i$ such that $j \neq \alpha , k \neq \beta $ but then their sum will yeild $x + y i$,  where $x,y \neq 0$ and so we don't get $0$ so we can say that our $w$ is unique.
\end{ver}

\begin{ver}
    multiplicative inverse
    Let $z \in \mathbb{C}$ so there exists some $\alpha, \beta \in \mathbb{R}$ so that $z = \alpha + \beta i$   let $w = \frac{\alpha}{\alpha^2 + \beta^2} + \frac{-\beta}{\alpha^2 + \beta^2} i$ 
    \begin{align*}
        zw &= \left( \alpha + \beta i \right)\left( \frac{\alpha}{\alpha^2 + \beta^2} + \frac{-\beta}{\alpha^2 + \beta^2} \right)i\\
           &= \frac{\alpha^2}{\alpha^2 + \beta^2} + \frac{\beta^2}{\alpha^2 + \beta^2} + \left( \frac{\alpha^2}{\alpha^2 + \beta^2} - \frac{\beta^2}{\alpha^2 + \beta^2} \right)\\
           &= 1 + 0i\\
           &=1
    \end{align*}
\end{ver}

\begin{ver}
    distributive property
    \begin{equation*}
        \lambda\left(w + z\right) = \lambda w + \lambda z \text{ for all  } \lambda, w, z \in \mathbb{C}    
    \end{equation*}
    \begin{align*}
        \lambda\left( w + z \right) &= \lambda \left( \alpha + \beta i + \delta + \varepsilon i \right) \\
                                    &= \lambda\left(\alpha + \delta + \left( \beta + \varepsilon \right)i\right) \\
                                    &= \lambda \alpha + \lambda \delta + \left( \lambda \beta + \lambda\varepsilon \right)i \\
                                    &= \lambda \alpha + \lambda \beta i + \lambda \delta + \lambda \varepsilon i \\
                                    &= \lambda \left( \alpha + \beta i \right) + \lambda \left( \delta + \varepsilon i \right) \\
                                    &= \lambda w + \lambda z
    \end{align*}
\end{ver}


% subsection complex_numbers (end)

% section chapter_1 (end)
\end{document}
