\documentclass[11pt]{book}

\RequirePackage{silence}
\WarningFilter{remreset}{The remreset package}

\title{MAT223 - Linear Algebra}
\author{Callum Cassidy-Nolan}

% Packages
\usepackage{amsmath}
\usepackage{amssymb}
\usepackage{mathtools}
\usepackage{xcolor}
\usepackage{amsthm}
\usepackage{thmtools}
\usepackage{amsfonts}
\usepackage{geometry}
\usepackage{gauss}
\usepackage{pifont}
\usepackage{hyperref}
\usepackage{witharrows}
\usepackage{cleveref}
\usepackage{tikz}
\usepackage{todonotes}
\usepackage{enumitem}
\usepackage{mdframed}
\usetikzlibrary{patterns,angles,quotes,decorations.pathreplacing}
\hypersetup{
    colorlinks=true, %set true if you want colored links
    linktoc=all,     %set to all if you want both sections and subsections linked
    linkcolor=blue,  %choose some color if you want links to stand out
}


% BlackSquare for proofs
\renewcommand{\qedsymbol}{$\blacksquare$}

\theoremstyle{definition}

% Matricies
\newcommand\mat[2][b]{\begin{#1matrix}#2\end{#1matrix}}

% Augmented matrix
\makeatletter
\renewcommand*\env@matrix[1][*\c@MaxMatrixCols c]{%
  \hskip -\arraycolsep
  \let\@ifnextchar\new@ifnextchar
  \array{#1}}
\makeatother

% Modular Arithmetic
\newcommand{\Mod}[1]{\ (\mathrm{mod}\ #1)}

% Automatic Parenthesis scaling
\delimitershortfall-1sp
\usepackage{mleftright}
\mleftright % make \left & \right behave like \mleft & \mright

% Theorems
\newtheoremstyle{break}
  {\topsep}{\topsep}%
  {\itshape}{}%
  {\bfseries}{}%
  {\newline}{}%
\theoremstyle{break}

\newtheorem{remark}{Remark}[section]

\newtheorem{ver}{Verfication}[section]

\newtheorem{ex}{Exercise}[section]

\newtheorem{eg}{Example}[section]

% definition env
\newmdtheoremenv{defn}{Definition}

% Note env
\newmdtheoremenv{nt}{Note}

% definition env no num
\newtheorem*{defnnonum}{Definition}

% theorem envs
\newtheorem{thm}{Theorem}

% theorem envs without counter

\newtheorem{propo}[thm]{Proposition}

\newtheorem{crly}[thm]{Corollary}

\newtheorem{lemma}[thm]{Lemma}

\newtheorem{axiom}[thm]{Axiom}

\newtheorem*{thmnonum}{Theorem}

\newtheorem*{propononum}{Proposition}

\newtheorem*{crlynonum}{Corollary}

\newtheorem*{lemmanonum}{Lemma}

\newtheorem*{axiomnonum}{Axiom}


\newtheorem{note}{Note}[section]

\newtheorem{mnote}[note]{Note}

\newtheorem*{notation}{Notation}
    % warning env
\newtheorem*{warning}{Warning}

% So todo's don't get cut off
\setlength{\marginparwidth}{3cm}

% Define cuboids
\tikzset{
  annotated cuboid/.pic={
    \tikzset{%
      every edge quotes/.append style={midway, auto},
      /cuboid/.cd,
      #1
    }
    \draw [every edge/.append style={pic actions, densely dashed, opacity=.5}, pic actions]
    (0,0,0) coordinate (o) -- ++(-\cubescale*\cubex,0,0) coordinate (a) -- ++(0,-\cubescale*\cubey,0) coordinate (b) edge coordinate [pos=1] (g) ++(0,0,-\cubescale*\cubez)  -- ++(\cubescale*\cubex,0,0) coordinate (c) -- cycle
    (o) -- ++(0,0,-\cubescale*\cubez) coordinate (d) -- ++(0,-\cubescale*\cubey,0) coordinate (e) edge (g) -- (c) -- cycle
    (o) -- (a) -- ++(0,0,-\cubescale*\cubez) coordinate (f) edge (g) -- (d) -- cycle;
    \path [every edge/.append style={pic actions, |-|}]
    (b) +(0,-5pt) coordinate (b1) edge ["\cubex \cubeunits"'] (b1 -| c)
    (b) +(-5pt,0) coordinate (b2) edge ["\cubey \cubeunits"] (b2 |- a)
    (c) +(3.5pt,-3.5pt) coordinate (c2) edge ["\cubez \cubeunits"'] ([xshift=3.5pt,yshift=-3.5pt]e)
    ;
  },
  /cuboid/.search also={/tikz},
  /cuboid/.cd,
  width/.store in=\cubex,
  height/.store in=\cubey,
  depth/.store in=\cubez,
  units/.store in=\cubeunits,
  scale/.store in=\cubescale,
  width=10,
  height=10,
  depth=10,
  units=cm,
  scale=.1,
}

% highlighting shortcuts
\newcommand{\hlimpo}[1]{\textcolor{red}{\textbf{#1}}}
\newcommand{\hlwarn}[1]{\textcolor{yellow}{\textbf{#1}}}
\newcommand{\hldefn}[1]{\textcolor{blue}{\index{#1}\textbf{#1}}}
\newcommand{\hlnotea}[1]{\textcolor{green}{\textbf{#1}}}
\newcommand{\hlnoteb}[1]{\textcolor{cyan}{\textbf{#1}}}
\newcommand{\hlb}[2]{\colorbox{#1!30!background}{#2}}
\newcommand{\hlbnotea}[1]{\hlb{green}{#1}}
\newcommand{\hlbnoteb}[1]{\hlb{cyan}{#1}}
\newcommand{\hlbnotec}[1]{\hlb{yellow}{#1}}
\newcommand{\hlbnoted}[1]{\hlb{magenta}{#1}}
\newcommand{\hlbnotee}[1]{\hlb{red}{#1}}
\newcommand{\WTP}{\textcolor{bwhite}{WTP} }
\newcommand{\WTS}{\textcolor{bwhite}{WTS} }



\begin{document}
\chapter*{Chapter 1}%
\label{chp:1}
% chapter 1

\subsection*{Complex Numbers}%
\label{sub:complex_numbers}
% subsection complex_numbers

\begin{ver}
    commutativity
    \begin{equation*}
        w + z = z + w \text{ and } wz = zw \text{ for all } w, z \in \mathbb{C}
    \end{equation*}
    We know that $w = \alpha + \beta i  \text{ and } z = j + k i  $ and so then 
    \begin{equation*}
        w + z = \alpha + \beta i + j + k i = \left( \alpha + j \right) + \left( \beta +k \right)i = \left( j + \alpha  \right) + \left( k + \beta  \right) = z + w
    \end{equation*}
    also from the commutivity of the Real Numbers it follows that 
    \[
        w \cdot z= \left( \alpha + \beta i  \right)  \cdot \left( j + k i  \right) = \left( \alpha j  - \beta k \right)  + \left( \alpha k + \beta j \right) i= \left( j\alpha  - k\beta  \right)  + \left( k\alpha  + j\beta  \right) = z \cdot w;
    \]
    
\end{ver}

\begin{ver}
    associativity 
    \begin{equation*}
        \left( z_1 + z_2 \right) + z_3 = z_1 + \left( z_2 + z_3 \right) \text{ and } \left( z_1 z_2 \right)z_3 = z_1\left( z_2z_3 \right) \text{ for all } z_1, z_2, z_3 \in \mathbb{C}
    \end{equation*}
    We know $z_1 = \alpha + \beta i, z_2 = j + k i, z_3 = l + m i$ and thus 
    \begin{align*}
        \left( z_1 + z_2 \right) + z_3 &= \alpha  + j + \left( \beta  + k \right) i + l + m i \\
                                       &= \left( \alpha + j + l \right) + \left( \beta + k + m \right) i \\
                                       &= z_1 + \left( z_2 + z_3 \right) 
    \end{align*}

    Now we'll do multiplication, we know
    \begin{align*}
        \left( z_1z_2 \right)z_3 &= \left(   \alpha  \cdot j - \beta  \cdot k + \left( \alpha k + \beta j \right) i \right) \left( l + m i \right) \\
                                 &= ajl - blk - \left( \alpha km + \beta jm \right) + \left( \alpha kl + \beta jl + ajm \right) i
    \end{align*}
    Now we also have
    \begin{align*}
        z_1\left( z_2z_3 \right) &= \alpha + \beta i\left( jl - km + \left( kl + jm \right)i \right)\\
                                 &= ajl - akm - \left( \beta kl + \beta jm \right) + \left( \beta jl + akm + ajm \right) i
    \end{align*}
    And due to the associativity of $\mathbb{R}$ then we can say $\left( z_1z_2 \right)z_3 = z_1 \left( z_2z_3 \right)$ 
\end{ver}

\begin{ver}
    identities
    \begin{equation*}
        z + 0 = z \text{ and } z1 = z \text{ for all } z \in \mathbb{C}
    \end{equation*}
    We know that 
    \begin{align*}
        z + 0 &= \alpha + \beta i + 0 + 0 i\\
              &= \left( \alpha + 0 \right) + \left( \beta + 0 \right) i \\
              &= \alpha + \beta i\\
              &= z  
    \end{align*}
    For multiplication we have
    \begin{align*}
        z1 &=\alpha + \beta i\left( 1 + 0i \right)\\
           &= \alpha - \beta 0 + \left( \beta + 0\alpha  \right)\\
           &= \alpha + \beta i\\
    \end{align*}
\end{ver}

\begin{ver}
    additive inverse\\
    for every $z \in \mathbb{C}$ there is a unique $w \in \mathbb{C}$ such that $z + w = 0$ 

    Let $z \in \mathbb{C}$ and so $z = \alpha + \beta i$ now we'll take $w = -\alpha  + -\beta  i$ 
    \begin{align*}
        z + w &= \alpha + \beta i + -\alpha  + -\beta  i\\
              &= \left( \alpha -\alpha  \right) + \left( \beta - \beta  \right)i\\
              &=0 + 0 i\\
              &=0
    \end{align*}
    To show that our choice of $w$ was unique assume there is another solution namely $w = j + k i$ such that $j \neq \alpha , k \neq \beta $ but then their sum will yeild $x + y i$,  where $x,y \neq 0$ and so we don't get $0$ so we can say that our $w$ is unique.
\end{ver}

\begin{ver}
    multiplicative inverse
    Let $z \in \mathbb{C}$ so there exists some $\alpha, \beta \in \mathbb{R}$ so that $z = \alpha + \beta i$   let $w = \frac{\alpha}{\alpha^2 + \beta^2} + \frac{-\beta}{\alpha^2 + \beta^2} i$ 
    \begin{align*}
        zw &= \left( \alpha + \beta i \right)\left( \frac{\alpha}{\alpha^2 + \beta^2} + \frac{-\beta}{\alpha^2 + \beta^2}i \right)\\
           &= \frac{\alpha^2}{\alpha^2 + \beta^2} + \frac{\beta^2}{\alpha^2 + \beta^2} + \left( \frac{\alpha \beta }{\alpha^2 + \beta^2} - \frac{\alpha \beta }{\alpha^2 + \beta^2} \right)i\\
           &= 1 + 0i\\
           &=1
    \end{align*}
\end{ver}

\begin{ver}
    distributive property
    \begin{equation*}
        \lambda\left(w + z\right) = \lambda w + \lambda z \text{ for all  } \lambda, w, z \in \mathbb{C}    
    \end{equation*}
    \begin{align*}
        \lambda\left( w + z \right) &= \lambda \left( \alpha + \beta i + \delta + \varepsilon i \right) \\
                                    &= \lambda\left(\alpha + \delta + \left( \beta + \varepsilon \right)i\right) \\
                                    &= \lambda \alpha + \lambda \delta + \left( \lambda \beta + \lambda\varepsilon \right)i \\
                                    &= \lambda \alpha + \lambda \beta i + \lambda \delta + \lambda \varepsilon i \\
                                    &= \lambda \left( \alpha + \beta i \right) + \lambda \left( \delta + \varepsilon i \right) \\
                                    &= \lambda w + \lambda z
    \end{align*}
\end{ver}

\begin{ver}
    $\mathbb{F}   ^{n} $ is a vector space over $\mathbb{F}  $  we know 
    \[
        \mathbb{F} ^{n} = \left\{ \left( x_{1} , x_{2} , \dotsc  , x_{n - 1} , x_{n}  \right) : x_{j} \in \mathbb{F} \text{ for } j= 1, 2,  \ldots,  n-1, n  \right\} 
    \]
    Let $u, v \in \mathbb{F}^{n} $ 
    \begin{itemize}
        \item commutivity: we will show 
            \[
            u + v= v + u
            \]
            we have 
            \begin{align*}
                \left( u_{1} , u_{2} , \dotsc  , u_{n - 1} , u_{n}  \right)  + \left( v_{1} , v_{2} , \dotsc  , v_{n - 1} , v_{n}  \right) &= \left( u_{1} + v_{1} , \ldots  , u_{n} + v_{n}  \right) \\
                \intertext{But since we have both commutivity for $\mathbb{R} $ and $\mathbb{C} $ it follows that }
                                                                                                                                           &= \left( v_{1} + u_{1} ,\ldots , v_{n} + u_{n}  \right)   \\ 
                                                                                                                                           &= v + u 
            \end{align*}
        \item associativity also follows similarly from the associativity of $\mathbb{R} \text{ and } \mathbb{C} $ 
        \item the additive identity is $0 = \left( 0, 0,  \ldots,  0, 0 \right) $ we will prove $u + 0= u$. As we have the additive identity for both $\mathbb{R} \text{ and } \mathbb{C} $ 
            \begin{align*}
                \left( u_{1} , u_{2} , \dotsc  , u_{n - 1} , u_{n}  \right)  + 0 &=  \left( 0 + u_{1} , 0 + u_{2} , \dotsc  , 0 + u_{n - 1} , 0 + u_{n}  \right) \\ 
                &=u 
            \end{align*}
        \item additive inverse: we have found an addtive inverse for the Real Numbers and complex numbers so if we are dealing with $\mathcal{F} $ we know there exists a $w \in \mathbb{F}^{n}  $ where $w$ is a list of additive inverses for each $u_{1} , u_{2} , \dotsc  , u_{n - 1} , u_{n} $ 
            \[
                u  + w= \left( u_{1} + w_{1} , u_{2} + w_{2} , \dotsc  , u_{n - 1} + w_{n - 1} , u_{n} + w_{n}  \right) = 0
            \]
        \item we know that the multiplicative identity of multiplying by 1, holds in the Real Numbers and complex numbers so we know 
            \[
                1\left( u \right) = \left( 1 \cdot u_{1} , 1 \cdot u_{2} , \dotsc  , 1 \cdot u_{n - 1} , 1 \cdot u_{n}  \right) = \left( u_{1} , u_{2} , \dotsc  , u_{n - 1} , u_{n}  \right) = u
            \]
        \item distributive property 
            \begin{align*}
                a\left( u + v \right) &=  \left( a \cdot (u_{1} + v_{1} ), a \cdot (u_{2} + v_{2} ), \dotsc  , a \cdot (u_{n - 1} + v_{n - 1} ), a \cdot (u_{n} + v_{n}  \right) \\
                                      &= \left( a \cdot u_{1} + a \cdot v_{1}  a \cdot u_{2} + a \cdot v_{2}  \dotsc   a \cdot u_{n - 1} + a \cdot v_{n - 1}  a \cdot u_{n} + a \cdot v_{n}  \right)   \\ 
            \end{align*}
    \end{itemize}
\end{ver}

% subsection complex_numbers (end)

% section chapter_1 (end)


% chapter 1 (end)

\end{document}
