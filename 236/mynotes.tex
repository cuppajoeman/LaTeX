\documentclass[11pt]{book}

\RequirePackage{silence}
\WarningFilter{remreset}{The remreset package}

\title{MAT236 - Intro to the Theory of Computation}
\author{Callum Cassidy-Nolan}

% Packages
\usepackage{amsmath}
\usepackage{amssymb}
\usepackage{mathtools}
\usepackage{xcolor}
\usepackage{amsthm}
\usepackage{thmtools}
\usepackage{amsfonts}
\usepackage{geometry}
\usepackage{gauss}
\usepackage{pifont}
\usepackage{hyperref}
\usepackage{witharrows}
\usepackage{cleveref}
\usepackage{tikz}
\usepackage{bm}
\usepackage{todonotes}
\usepackage{enumitem}
\usepackage{mdframed}
\usetikzlibrary{patterns,angles,quotes,decorations.pathreplacing}
\hypersetup{
    colorlinks=true, %set true if you want colored links
    linktoc=all,     %set to all if you want both sections and subsections linked
    linkcolor=blue,  %choose some color if you want links to stand out
}


% BlackSquare for proofs
\renewcommand{\qedsymbol}{$\blacksquare$}

\theoremstyle{definition}

% Matricies
\newcommand\mat[2][b]{\begin{#1matrix}#2\end{#1matrix}}

% Augmented matrix
\makeatletter
\renewcommand*\env@matrix[1][*\c@MaxMatrixCols c]{%
  \hskip -\arraycolsep
  \let\@ifnextchar\new@ifnextchar
  \array{#1}}
\makeatother

% Modular Arithmetic
\newcommand{\Mod}[1]{\ (\mathrm{mod}\ #1)}

% Automatic Parenthesis scaling
\delimitershortfall-1sp
\usepackage{mleftright}
\mleftright % make \left & \right behave like \mleft & \mright

% Theorems
\newtheoremstyle{break}
  {\topsep}{\topsep}%
  {\itshape}{}%
  {\bfseries}{}%
  {\newline}{}%

\theoremstyle{break}

\newtheorem{remark}{Remark}[section]

\newtheorem{ver}{Verfication}[section]

\newtheorem{ex}{Exercise}[section]

\newtheorem{eg}{Example}[section]

% definition env
\newmdtheoremenv{defn}{Definition}

% Note env
\newmdtheoremenv{nt}{Note}

% definition env no num
\newtheorem*{defnnonum}{Definition}

% theorem envs
\newtheorem{thm}{Theorem}

% theorem envs without counter

\newtheorem{propo}[thm]{Proposition}

\newtheorem{crly}[thm]{Corollary}

\newtheorem{lemma}[thm]{Lemma}

\newtheorem{axiom}[thm]{Axiom}

\newtheorem*{thmnonum}{Theorem}

\newtheorem*{propononum}{Proposition}

\newtheorem*{crlynonum}{Corollary}

\newtheorem*{lemmanonum}{Lemma}

\newtheorem*{axiomnonum}{Axiom}


\newtheorem{note}{Note}[section]

\newtheorem{mnote}[note]{Note}

\newtheorem*{notation}{Notation}
    % warning env
\newtheorem*{warning}{Warning}

% So todo's don't get cut off
\setlength{\marginparwidth}{3cm}

% Define cuboids
\tikzset{
  annotated cuboid/.pic={
    \tikzset{%
      every edge quotes/.append style={midway, auto},
      /cuboid/.cd,
      #1
    }
    \draw [every edge/.append style={pic actions, densely dashed, opacity=.5}, pic actions]
    (0,0,0) coordinate (o) -- ++(-\cubescale*\cubex,0,0) coordinate (a) -- ++(0,-\cubescale*\cubey,0) coordinate (b) edge coordinate [pos=1] (g) ++(0,0,-\cubescale*\cubez)  -- ++(\cubescale*\cubex,0,0) coordinate (c) -- cycle
    (o) -- ++(0,0,-\cubescale*\cubez) coordinate (d) -- ++(0,-\cubescale*\cubey,0) coordinate (e) edge (g) -- (c) -- cycle
    (o) -- (a) -- ++(0,0,-\cubescale*\cubez) coordinate (f) edge (g) -- (d) -- cycle;
    \path [every edge/.append style={pic actions, |-|}]
    (b) +(0,-5pt) coordinate (b1) edge ["\cubex \cubeunits"'] (b1 -| c)
    (b) +(-5pt,0) coordinate (b2) edge ["\cubey \cubeunits"] (b2 |- a)
    (c) +(3.5pt,-3.5pt) coordinate (c2) edge ["\cubez \cubeunits"'] ([xshift=3.5pt,yshift=-3.5pt]e)
    ;
  },
  /cuboid/.search also={/tikz},
  /cuboid/.cd,
  width/.store in=\cubex,
  height/.store in=\cubey,
  depth/.store in=\cubez,
  units/.store in=\cubeunits,
  scale/.store in=\cubescale,
  width=10,
  height=10,
  depth=10,
  units=cm,
  scale=.1,
}

% highlighting shortcuts
\newcommand{\hlimpo}[1]{\textcolor{red}{\textbf{#1}}}
\newcommand{\hlwarn}[1]{\textcolor{yellow}{\textbf{#1}}}
\newcommand{\hldefn}[1]{\textcolor{blue}{\index{#1}\textbf{#1}}}
\newcommand{\hlnotea}[1]{\textcolor{green}{\textbf{#1}}}
\newcommand{\hlnoteb}[1]{\textcolor{cyan}{\textbf{#1}}}
\newcommand{\hlb}[2]{\colorbox{#1!30!background}{#2}}
\newcommand{\hlbnotea}[1]{\hlb{green}{#1}}
\newcommand{\hlbnoteb}[1]{\hlb{cyan}{#1}}
\newcommand{\hlbnotec}[1]{\hlb{yellow}{#1}}
\newcommand{\hlbnoted}[1]{\hlb{magenta}{#1}}
\newcommand{\hlbnotee}[1]{\hlb{red}{#1}}
\newcommand{\WTP}{\textcolor{bwhite}{WTP} }
\newcommand{\WTS}{\textcolor{bwhite}{WTS} }



\begin{document}

\maketitle

\tableofcontents

\renewcommand{\listtheoremname}{List of Definitions}
\listoftheorems[ignoreall,show={defn}]


\renewcommand{\listtheoremname}{\textsl{List of Theorems}}
\listoftheorems[ignoreall,
show={axiom,lemma,thm,crly,propo}
]


\chapter{Lecture 1}%
\label{chp:lecture_1}
% chapter lecture_1

\begin{defn}[Predicate]\index{Predicate}\label{defn:predicate}
    defined over some variable and denoted by a statement about a set of elements.
    \paragraph{Example}
    \[
        P(n): `` n \text{ is an odd natural number }  "\text{, where } n \in \mathbb{N} 
    \]
\end{defn}

\section{Simple Induction}%
\label{sec:simple_induction}
% section simple_induction

Allows us to prove a predicate P holds for all natural numbers greater than or equal to $b\in \mathbb{N} $ that is 
\[
\forall n \in \mathbb{N}, n \ge b \implies P\left(n\right) 
\]

The principle behind it is this
\begin{itemize}
    \item If $P\left(b\right) $ holds , and we have $\forall k \in \mathbb{N} , k \ge b, P\left(k\right) \implies P\left(k + 1\right) $ 
    \item Then $\forall n \in \mathbb{N} , n \ge b \implies P\left(n\right) $ holds 
\end{itemize}

\underline{Justification, Informal}
\begin{itemize}
    \item suppose $P\left(b\right)  $  hold (base case) 
    \item suppose $\forall k \in \mathbb{N} , k \ge b, P\left(k\right) \implies P\left(k + 1\right) $ (Induction Step) 
\end{itemize}

Then 
\[
P\left(b\right) \implies P\left(b + 1\right) \implies \ldots 
\]

Prove the following 
\[
    \forall n \in \mathbb{N} , \sum_{i=0}^{n} = \frac{n\left( n + 1 \right) }{2} 
\]

\begin{proof}
    $ $\newline
    We define the following predicate 
    \[
        P(n): `` \sum_{i=0}^{n} i = \frac{n\left( n + 1 \right) }{2} "\text{, where } n \in \mathbb{N} 
    \]

    
\end{proof}
\begin{proof}
    $ $\newline Be begin
    \begin{itemize}
        \item \textbf{Base Case:}\\
            We show that $P(0)$ holds, we know 
            \[
                \sum_{i=0}^{0} = \frac{0\left( 0 + 1 \right) }{2} 
            \]
            therefore right hand side equals left hand side, so we have $P\left(0\right) $ holds .
        \item \textbf{Induction Step:}\\
            Let $k \in \mathbb{N}$ and assume that $P(k)$ holds, that is 
            \[
                \sum_{i=0}^{k} i = \frac{k\left( k + 1 \right) }{2} 
            \]
            
            We'll show that $P(k + 1)$ holds, so we must show that 
            \[
                \sum_{i=0}^{k + 1} i = \frac{\left( k + 1 \right) \left( k + 2 \right) }{2} 
            \]
            Intuition: We need to get to our IH, let's break the summation down, we know 
            \begin{align*}
                \sum_{i=0}^{k + 1} i &= \sum_{i=0}^{k} i  + \left( k + 1 \right)   \\
                &= \frac{k\left( k + 1 \right) }{2} + \left( k + 1 \right)   \\ 
                &= \frac{k + 1\left( k + 2 \right) }{2} 
            \end{align*}
    \end{itemize}
    Thus by the principle of induction we have proven the original statement.
\end{proof}

\paragraph{Summary}
\begin{enumerate}
    \item Define the predicate 
    \item Prove that the base case holds 
    \item Induction Step: Assume the Induction Hypothesis, show what we will prove.
    \item Use Induction Hypothesis to prove $P\left(k + 1\right) $ 
\end{enumerate}

% subsection  (end)

We will prove 
\[
a_{n} = 2^{n + 1}  - 1
\]

Where $a_{0} = 1, a_{n} = 2a_{ - 1}  + 1$ for $n\ge 1$  

\begin{proof}
    $ $\newline
    We define the following predicate
    \[
    P(n): `` a_{n} = 2^{n + 1}  - 1 "\text{, where } n \in \mathbb{N} 
    \]
    \begin{itemize}
        \item \textbf{Base Case:}\\
            We show that $P(0)$ holds, we know 
            \[
            a_{0} = 2^{1}  - 1
            \]
            therefore $P\left(0\right) $ holds 
        \item \textbf{Induction Step:}\\
            Let $k \in \mathbb{N}$ and assume that $P(k)$ holds, that is 
            \[
            a_{k} = 2^{k + 1}  - 1
            \]
            
            We'll show that $P(k + 1)$ holds, so we must show that 
            \[
            a_{k + 1} = 2^{k + 2}  - 1
            \]
            Intuition: we use the recurvie definition to access the Induction Hypothesis, we know 
            \begin{align*}
                a_{k + 1} &= 2a_{k}  + 1  \\ 
                &= 2\left( 2^{k + 1}  - 1 \right)  + 1  \\ 
                &= 2^{k + 2}  - 2  + 1\\
                &= 2^{k + 2}  - 1 
            \end{align*}
    \end{itemize}
    Thus by the principle of induction we have proven the original statement.
\end{proof}

\paragraph{Homework Quesiton}
We will prove 
\[
\forall n \in \mathbb{N} , n > 4 \implies 2^{n} > n^2 
\]

\begin{proof}
    $ $\newline
    We define the following predicate
    \[
    P(n): `` 2^{n} > n^2  "\text{, where } n \in \mathbb{N} 
    \]
    \begin{itemize}
        \item \textbf{Base Case:}\\
            We show that $P(5)$ holds, we know 
            \[
            2^{5} > 5^2 \Leftrightarrow 32 > 25
            \]
            therefore $P\left(5\right) $ holds .
        \item \textbf{Induction Step:}\\
            Let $k \in \mathbb{N}$ and assume that $P(k)$ holds, that is 
            \[
            2^{k} > k^2 
            \]
            
            We'll show that $P(k + 1)$ holds, so we must show that 
            \[
                2^{k + 1} > \left( k + 1 \right) ^2 
            \]
            We know 
            \[
                2^{k} 2 > 2k^2  \\ 
            \]
            At this point if we can show that $2k^2 \ge \left( k + 1 \right) ^2 $ we are done, let's see if it's true
            \begin{gather*}
                k > 5 > 4\\
                \left( k - 1 \right) ^2 > 9 > 2\\
                \left( k - 1 \right) ^2 > 2\\
                k^2  - 2k + 1 > 2\\
                k^2 > 2k  + 1\\
                2k^2 > \left( k + 1 \right) ^2 
            \end{gather*}
            So it is true, therefore we can say 
            \[
                2^{k + 1} > \left( k + 1 \right) ^2 
            \]
            as required. 
    \end{itemize}
    Thus by the principle of induction we have proven the original statement.
\end{proof}


% section simple_induction (end)

% chapter lecture_1 (end)

\chapter{Lecture 2}%
\label{chp:lecture_2}
% chapter lecture_2

\section{Simple Induction Downfall}%
\label{sec:simple_induction_downfall}
% section simple_induction_downfall

Observe if we are attempting to prove a statement, that requires us to know something more than just a property of the previous element we are proving over, for example a proof of prime factorization requires, us to know about all previous primes, this will become clear soon.

We will prove 
\[
\forall n \in \mathbb{N} , n \ge 2 \implies n \text{ has a prime factorization  } 
\]

In our Induction Step we observe that we will have $k + 1 = a \cdot b$ where $a, b$ are definelty not both $k + 1$,  therefore , we will not be able to conclude anything about both of them, and we cannot proceed , therefore we need a stronger hypothesis.

\begin{defn}[Complete Induction]\index{Complete Induction}\label{defn:complete_induction}
    It is similar to Induction, with the following modifications
    \begin{itemize}
        \item Base Case : show $P\left(b\right) $ holds 
        \item Induction Step: let $k\in \mathbb{N} , k\ge b$ we assume the following our (Induction Hypothesis )
            \[
            P\left(b\right) , P\left(b + 1\right) , \ldots , P\left(k\right) 
            \]
            Or equivalently, for $k\ge b$ 
            \[
            \forall j \in \mathbb{N} , b \le j \le k, P\left(j\right) 
            \]
    \end{itemize}
\end{defn}

We will prove the statement we tried earlier with simple induction.

\begin{proof}
    $ $\newline
    We define the following predicate
    \[
    P(n): `` n \text{ has a prime factorization  }  "\text{, where } n \in 
    \]
    note, formally this means
    \[
    n = \prod_{i=1}^{n} q_{i}  
    \]
    where $q_{k} \in \mathbb{N} \text{ and } $ is prime .
    \begin{itemize}
        \item \textbf{Base Case:}\\
            We show that $P(2)$ holds, we know that $2$ is a prime number , therefore it is itself a product of primes, as required.
            
        \item \textbf{Induction Step:}\\
            Let $k \in \mathbb{N} \text{ such that } k\ge 2$ and assume that $\forall j \in \mathbb{N} , 2 \le j \le k, P\left(j\right) $ holds, that is $j$ has a prime factorization.
            We'll show that $P(k + 1)$ holds, so we must show that $j$ has  a prime factorization.
            \begin{itemize}
                \item \textbf{Case 1}: $j$ is prime, then we are done as it is already a product of primes.
                \item \textbf{Case 2}: $j$ is composite, that is there exists an $a, b \in \mathbb{N} \text{ such that }, j= ab $ also $  a, b \neq 1 \land a, b \neq j $, we know that $a \text{ and } b$ cannot exceed j, or else $ab > j$ a contradiction same goes for $b$. Therefore $1 < a,b < j$ so by our Induction Hypothesis , they have a prime factorization. And thus we take the product of these two product of primes and $j$ is then a product of primes.
            \end{itemize}
    \end{itemize}
    Thus by the principle of complete induction we have proven the original statement.
\end{proof}

\begin{note}
observe in the above proof that an intricate part of the proof was to show that $a, b$ where in fact in the correct range of the Induction Hypothesis, this is very important.
\end{note}

% section simple_induction_downfall (end)

\section{Common Problems in Complete Induction}%
\label{sec:common_problems_in_complete_induction}
% section common_problems_in_complete_induction

We will prove 
\[
\forall n \in \mathbb{N} , n\ge 1, a_{n} = 2f_{n}  - 1
\]
Where 
\[
    f_{n}  = 
    \begin{cases}
        1, \text{ if } n= 1\\
        1, \text{ if } n= 2\\
        \text{ otherwise  }, f_{n - 1}  + f_{n  - 2} 
    \end{cases}
\]
\[
a_{n} = 
\begin{cases}
    1, \text{ if } n = 1\\
    2, \text{ if } n= 2 \\
    \text{ otherwise  } a_{n  - 1}  + a_{n - 2}  + 1

\end{cases}
\]

\begin{proof}
    $ $\newline
    We define the following predicate
    \[
    P(n): `` a_{n} = 2f_{n}  - 1 "\text{, where } n \in \mathbb{N} 
    \]
    \begin{itemize}
        \item \textbf{Base Case:}\\
            We show that $P(1)$ holds, we know 
            \[
                a_{1} = 2f_{1} -1 \Leftrightarrow 1= 2 - 1
            \]
            therefore $P\left(1\right) $ holds 
            \begin{itemize}
                \item \underline{Additional Base Case Required!} 
                \item we show that $P\left(2\right) $ holds ,that is 
                    \[
                    1 = 2  - 1 \Leftrightarrow a_{2} = 2f_{2}  - 1
                    \]
            \end{itemize}
        \item \textbf{Induction Step:}\\
            Let $k \in \mathbb{N}, k \ge \cancelto{2}{1}$ and assume that $\forall j \in \mathbb{N} , 1 \le  j \le k, P\left(j\right)$ holds, that is 
            \[
                a_{j} = 2f_{j}  - 1
            \]
            We'll show that $P(k + 1)$ holds, so we must show that 
            \[
            a_{k + 1} = 2f_{k + 1}  - 1
            \]
            \begin{itemize}
                \item \textbf{Case 1}: $k + 1 \ge 3$,  if so we have
                    \begin{align*}
                        a_{k + 1} &= a_{k}  + a_{k - 1}  + 1  \tag{$\alpha$ }\\ 
                        &= 2f_{k}  - 1 + \left( 2f_{k - 1}  - 1 \right)  + 1\\
                        &= 2f_{k}  + 2f_{k - 1}  - 1  \\ 
                        &= 2\left( f_{k}  + f_{k - 1}  \right)  - 1 \\
                        &= 2f_{k + 1} -1 
                    \end{align*}
                \item \textbf{Case 2}: $k + 1 \le 2 \Leftrightarrow k = 1$
                    \begin{align*}
                        a_{2} = 2f_{2}  - 1 \Leftrightarrow 1= 1
                    \end{align*}
            \end{itemize}
            We know 
    \end{itemize}
    Thus by the principle of complete induction we have proven the original statement.
\end{proof}

The above proof may have seemed, fine, though there is a problem. $k \ge 1$ so $k - 1 \ge 0$, though we we only have information about $P\left(j\right) , \text{ for } j \in \left\{ 1, \ldots , k \right\} $. Therefore we necessarily require, $k - 1 \ge 1$ so that it is contained within the Induction Hypothesis bounds, that is true if and only if $k \ge 2$, so we add an additional Base Case to cover the gap we created.

\subsection{Looking for Flaws in Strong Induction}%
\label{sub:looking_for_flaws_in_strong_induction}
% subsection looking_for_flaws_in_strong_induction

When looking for flaws, whenever the Induction Hypothesis is used, you must verify that in fact is in the correct range. If we have a proof by induction and our Induction Hypothesis assumed over $0, \ldots k$ and in our algebra we have used the Induction Hypothesis on $k  - 1$,  then we know this will cause problems, so then we must say $k  - 1 \ge 0 \Leftrightarrow k \ge 1$ and thus we must fill in the extra base case, usually if the statement is false, then this is what causes a contradiction.

% subsection looking_for_flaws_in_strong_induction (end)

% section common_problems_in_complete_induction (end)

\subsection{Induction on Different Sets}%
\label{sub:induction_on_different_sets}
% subsection induction_on_different_sets

\begin{itemize}
    \item We can induct on the set of even natural numbers, say we prove $P\left(0\right) $,  and also $P\left(k\right) \implies P\left(k + 2\right) $ then we get all even numbers.
    \item $P\left(0\right) $ and we show the following 
        \begin{align*}
            P\left(k\right) \implies P\left(k + 1\right)  && P\left(k\right) \implies P\left(k - 1\right) 
        \end{align*}
    \item Recall that there is no smallest rational number, and that the sum of two rational numbers is also a rational number, therefore there is no smallest increment that we can induct on, so I believe we cannot induct over $\mathbb{Q} $ 
\end{itemize}

% subsection induction_on_different_sets (end)

% chapter lecture_2 (end)

\chapter{Preliminaries}%
\label{chp:preliminaries}
% chapter preliminaries

\section{Sets}%
\label{sec:sets}
% subsection sets

\begin{itemize}
    \item We now have the concept of the cardinality of a set being infinity that is for a set $A$ we have, $\left| A \right| = \infty $
    \item $\infty $: for all integers $k$ we have $k < \infty $ 
    \item Describing by listing all elements explicitly is called an \underline{extensional}  description, we can state the property that characterises it's elements, then we have an \underline{internal} description (set-builder).
    \item A \underline{proper}  sub/super $A \subset B$ set means that every element of $A$ is also an element of $B$ moreover, there is at least one element of $B$ that is not in $A$.   
    \item If $A \cap  B = \emptyset  $ , that is $A \text{ and } B$ have nothing in common, then we say they are \underline{disjoint}.
    \item The difference of $A - B$ is the set of elements in $A$, that don't belong to $B$ 
    \item The intersection or union of an arbitary number, or infinite number of sets, is written as ($I$ is a set of indicies)
        \begin{gather*}
            \cup _{i \in I} A_{i} = \left\{ x: \text{ for some } i\in I, x \in A_{i}  \right\} \\
            \cap  _{i \in I} A_{i} = \left\{ x: \text{ for each } i\in I, x \in A_{i}  \right\} 
        \end{gather*}
    \item Partition: For a set $A$ a partition of $A$ is a set, that satisifies the following.
        \begin{itemize}
            \item $\mathcal{X} \subseteq \mathcal{P}(A) $ such that $X \in \mathcal{X}, \mathcal{X} \neq \emptyset $ 
        \item $X,Y \in \mathcal{X} \text{ such that } X\neq Y, X\cap Y= \emptyset $ and $\cup _{X\in \mathcal{X} } X= A$ 
        \end{itemize}
\end{itemize}

\subsection{Ordered Pairs}%
\label{sub:ordered_pairs}
% subsection ordered_pairs

\begin{itemize}
    \item Ordered pairs actually can be defined more primitively, we have $\left( a,b  \right) $ we can define it as the set $\left\{ \left\{ a \right\} , \left\{ a, b \right\}  \right\} $, the element that is of length 1, is viewed as the first element of the ordered set, the element of size two represents the ordered pair, one of which is the first, and the other is the second. 
    \item For example if we have $\left( j, k  \right) = \left( l, m \right)$, then $\left\{ \left\{ j \right\} , \left\{ j, k \right\}  \right\} = \left\{ \left\{ l \right\} , \left\{ l, m \right\}  \right\} $ therefore we must verify that they are subsets of eachother. 
\end{itemize}


% subsection ordered_pairs (end)


% subsection sets (end)

% chapter preliminaries (end)

\end{document}
