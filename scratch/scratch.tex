\documentclass[11pt]{article}

\usepackage{scratch} 


\begin{document}

Let \(S = \left\{ 1, 2, 4, 8, 16, 32 \right\} \). \\
Our main objective is to determine the total number of different sums that can be made with 3 distinct elements of \(S\) and to find the total number of these who are divisible by 4.

Consider the set \(X = \left\{ 1, 2, 3, 4 \right\} \) observe that \(2 + 3= 1 + 4\) so the the total number of different sums of X, is not simply \(\binom{4}{2} \). So we must verify that indeed each of our sums are unique.

We note that \(S\) consists of powers of two and we know that every binary number represents a unique decimal number, so indeed each sum is unique and therefore there are \(\binom{6}{3} \) of them. 

We will now determine which sums are divisible by 4. 


\end{document}
