\documentclass[11pt]{book}

\RequirePackage{silence}
\WarningFilter{remreset}{The remreset package}

\title{MAT247 - Probability with Computer Applications}
\author{Callum Cassidy-Nolan}

% Packages
\usepackage{amsmath}
\usepackage{amssymb}
\usepackage{mathtools}
\usepackage{xcolor}
\usepackage{amsthm}
\usepackage{thmtools}
\usepackage{amsfonts}
\usepackage{geometry}
\usepackage{gauss}
\usepackage{pifont}
\usepackage{hyperref}
\usepackage{witharrows}
\usepackage{cleveref}
\usepackage{tikz}
\usepackage{bm}
\usepackage{todonotes}
\usepackage{enumitem}
\usepackage{mdframed}
\usetikzlibrary{patterns,angles,quotes,decorations.pathreplacing}
\hypersetup{
    colorlinks=true, %set true if you want colored links
    linktoc=all,     %set to all if you want both sections and subsections linked
    linkcolor=blue,  %choose some color if you want links to stand out
}


% BlackSquare for proofs
\renewcommand{\qedsymbol}{$\blacksquare$}

\theoremstyle{definition}

% Matricies
\newcommand\mat[2][b]{\begin{#1matrix}#2\end{#1matrix}}

% Augmented matrix
\makeatletter
\renewcommand*\env@matrix[1][*\c@MaxMatrixCols c]{%
  \hskip -\arraycolsep
  \let\@ifnextchar\new@ifnextchar
  \array{#1}}
\makeatother

% Modular Arithmetic
\newcommand{\Mod}[1]{\ (\mathrm{mod}\ #1)}

% Automatic Parenthesis scaling
\delimitershortfall-1sp
\usepackage{mleftright}
\mleftright % make \left & \right behave like \mleft & \mright

% Theorems
\newtheoremstyle{break}
  {\topsep}{\topsep}%
  {\itshape}{}%
  {\bfseries}{}%
  {\newline}{}%

\theoremstyle{break}

\newtheorem{remark}{Remark}[section]

\newtheorem{ver}{Verfication}[section]

\newtheorem{ex}{Exercise}[section]

\newtheorem{eg}{Example}[section]

% definition env
\newmdtheoremenv{defn}{Definition}

% Note env
\newmdtheoremenv{nt}{Note}

% definition env no num
\newtheorem*{defnnonum}{Definition}

% theorem envs
\newtheorem{thm}{Theorem}

% theorem envs without counter

\newtheorem{propo}[thm]{Proposition}

\newtheorem{crly}[thm]{Corollary}

\newtheorem{lemma}[thm]{Lemma}

\newtheorem{axiom}[thm]{Axiom}

\newtheorem*{thmnonum}{Theorem}

\newtheorem*{propononum}{Proposition}

\newtheorem*{crlynonum}{Corollary}

\newtheorem*{lemmanonum}{Lemma}

\newtheorem*{axiomnonum}{Axiom}


\newtheorem{note}{Note}[section]

\newtheorem{mnote}[note]{Note}

\newtheorem*{notation}{Notation}
    % warning env
\newtheorem*{warning}{Warning}

% So todo's don't get cut off
\setlength{\marginparwidth}{3cm}

% Define cuboids
\tikzset{
  annotated cuboid/.pic={
    \tikzset{%
      every edge quotes/.append style={midway, auto},
      /cuboid/.cd,
      #1
    }
    \draw [every edge/.append style={pic actions, densely dashed, opacity=.5}, pic actions]
    (0,0,0) coordinate (o) -- ++(-\cubescale*\cubex,0,0) coordinate (a) -- ++(0,-\cubescale*\cubey,0) coordinate (b) edge coordinate [pos=1] (g) ++(0,0,-\cubescale*\cubez)  -- ++(\cubescale*\cubex,0,0) coordinate (c) -- cycle
    (o) -- ++(0,0,-\cubescale*\cubez) coordinate (d) -- ++(0,-\cubescale*\cubey,0) coordinate (e) edge (g) -- (c) -- cycle
    (o) -- (a) -- ++(0,0,-\cubescale*\cubez) coordinate (f) edge (g) -- (d) -- cycle;
    \path [every edge/.append style={pic actions, |-|}]
    (b) +(0,-5pt) coordinate (b1) edge ["\cubex \cubeunits"'] (b1 -| c)
    (b) +(-5pt,0) coordinate (b2) edge ["\cubey \cubeunits"] (b2 |- a)
    (c) +(3.5pt,-3.5pt) coordinate (c2) edge ["\cubez \cubeunits"'] ([xshift=3.5pt,yshift=-3.5pt]e)
    ;
  },
  /cuboid/.search also={/tikz},
  /cuboid/.cd,
  width/.store in=\cubex,
  height/.store in=\cubey,
  depth/.store in=\cubez,
  units/.store in=\cubeunits,
  scale/.store in=\cubescale,
  width=10,
  height=10,
  depth=10,
  units=cm,
  scale=.1,
}

% highlighting shortcuts
\newcommand{\hlimpo}[1]{\textcolor{red}{\textbf{#1}}}
\newcommand{\hlwarn}[1]{\textcolor{yellow}{\textbf{#1}}}
\newcommand{\hldefn}[1]{\textcolor{blue}{\index{#1}\textbf{#1}}}
\newcommand{\hlnotea}[1]{\textcolor{green}{\textbf{#1}}}
\newcommand{\hlnoteb}[1]{\textcolor{cyan}{\textbf{#1}}}
\newcommand{\hlb}[2]{\colorbox{#1!30!background}{#2}}
\newcommand{\hlbnotea}[1]{\hlb{green}{#1}}
\newcommand{\hlbnoteb}[1]{\hlb{cyan}{#1}}
\newcommand{\hlbnotec}[1]{\hlb{yellow}{#1}}
\newcommand{\hlbnoted}[1]{\hlb{magenta}{#1}}
\newcommand{\hlbnotee}[1]{\hlb{red}{#1}}
\newcommand{\WTP}{\textcolor{bwhite}{WTP} }
\newcommand{\WTS}{\textcolor{bwhite}{WTS} }



\begin{document}

\maketitle

\tableofcontents

\renewcommand{\listtheoremname}{List of Definitions}
\listoftheorems[ignoreall,show={defn}]


\renewcommand{\listtheoremname}{\textsl{List of Theorems}}
\listoftheorems[ignoreall,
show={axiom,lemma,thm,crly,propo}
]


\chapter{Lecture 1}%
\label{chp:lecture_1}
% chapter lecture_1

\section{Stats vs Probability}%
\label{sec:stats_vs_probability}
% subsection stats_vs_probability

\underline{Statistics} 
\begin{itemize}
    \item Reverse Probability
    \item Making observations and then based on probability we describe the original model
    \item Critique whether a model is correct 
\end{itemize}
\underline{Probability} 
\begin{itemize}
    \item How likely based on a theoretical situation, for example a coin toss has a 50/50 chance of one of the outcomes.
    \item We know all the parameters.
\end{itemize}

% subsection stats_vs_probability (end)

\section{To do Well}%
\label{sec:to_do_well}
% section to_do_well

\begin{itemize}
    \item Understand the formulas
    \item Make sure to the hard questions
    \item Office hours in HS 386
\end{itemize}

% section to_do_well (end)

\newpage

\section{Useful Terminology}%
\label{sec:useful_terminology}
% section useful_terminology

\begin{defn}[Random Experiment]\index{Random Experiment}\label{defn:random_experiment}
    A process of gathering data or observations. We can perform the experiment multiple times so long as the conditions aren't changed and the outcome of each experiment is random, we don't know what the result will be, though we know the set of possible outcomes. 
\end{defn}

\underline{Examples} 

\begin{itemize}
    \item Rolling a die, and observing the number on the top face. 
    \item Rolling $n$ dice and observing the resulting pair of numbers. 
    \item Drawing 3 cards from a deck of cards
    \item Asking a professor how old they are
\end{itemize}

\begin{defn}[Sample Space]\index{Sample Space}\label{defn:sample_space}
    This is the set of all possible outcomes/results from a random experiment , we denote this set as
    \[
    \Omega \text{ or } S
    \]
    The sample space depends on the outcome of interest.
\end{defn}

\begin{nt}
    If our outcome of interest is say, whether after flipping a coin it is heads or tails, we don't care how many times it turned in the air before landing on the ground.
\end{nt}

\begin{enumerate}
    \item [\textbf{Examples}] 
    \item $S = \left\{ n \in \mathbb{N} : 1 \le  n \le 20 \right\} $.  All the faces of the dice.
    \item $\Omega _{1} = \left\{ \text{ True } , \text{ False } \right\} $ 
    \item $\Omega _{2} = \mathbb{R} ^{\ge 0} $. If they are precise in their measurement, though more likely to any hour between 0 and 10. 
\end{enumerate}

\begin{defn}[Event]\index{Event}\label{defn:event}
    Any subset of our sample space of interest. 
    \begin{itemize}
        \item Simple Event: One with exactly one outcome
        \item Compound Event: One with multiple outcomes 
    \end{itemize}
\end{defn}

    \begin{itemize}
        \item [\textbf{Simple Event}] 
        \item flipping a coin, and observing whether it has landed on it's edge (that's possible!)
        \item [\textbf{Compound Event}] 
        \item flipping 3 coins, and observing whether at least one has landed on it's edge. 
    \end{itemize}

\begin{defn}[Complement]\index{Complement}\label{defn:complement}
    The \underline{complement} of an event A is the event consisting of outcome that are nto in A. We denote this as $A^{c} $ .\\
    \textbf{For example:} the complement of our previous example, that would be if no, coin has landed on it's edge.
\end{defn}

% section useful_terminology (end)

% chapter lecture_1 (end)


\end{document}
