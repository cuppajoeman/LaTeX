\documentclass[11pt]{book}

\RequirePackage{silence}
\WarningFilter{remreset}{The remreset package}

\title{MAT237 - Multi-variable Calculus}
\author{Callum Cassidy-Nolan}

% Packages
\usepackage{amsmath}
\usepackage{amssymb}
\usepackage{mathtools}
\usepackage{xcolor}
\usepackage{amsthm}
\usepackage{thmtools}
\usepackage{amsfonts}
\usepackage{geometry}
\usepackage{gauss}
\usepackage{pifont}
\usepackage{hyperref}
\usepackage{witharrows}
\usepackage{cleveref}
\usepackage{tikz}
\usepackage{todonotes}
\usepackage{enumitem}
\usepackage{mdframed}
\usetikzlibrary{patterns,angles,quotes,decorations.pathreplacing}
\hypersetup{
    colorlinks=true, %set true if you want colored links
    linktoc=all,     %set to all if you want both sections and subsections linked
    linkcolor=blue,  %choose some color if you want links to stand out
}


% BlackSquare for proofs
\renewcommand{\qedsymbol}{$\blacksquare$}

\theoremstyle{definition}

% Matricies
\newcommand\mat[2][b]{\begin{#1matrix}#2\end{#1matrix}}

% Augmented matrix
\makeatletter
\renewcommand*\env@matrix[1][*\c@MaxMatrixCols c]{%
  \hskip -\arraycolsep
  \let\@ifnextchar\new@ifnextchar
  \array{#1}}
\makeatother

% Modular Arithmetic
\newcommand{\Mod}[1]{\ (\mathrm{mod}\ #1)}

% Automatic Parenthesis scaling
\delimitershortfall-1sp
\usepackage{mleftright}
\mleftright % make \left & \right behave like \mleft & \mright

% Theorems
\newtheoremstyle{break}
  {\topsep}{\topsep}%
  {\itshape}{}%
  {\bfseries}{}%
  {\newline}{}%
\theoremstyle{break}

\newtheorem{remark}{Remark}[section]

\newtheorem{ver}{Verfication}[section]

\newtheorem{ex}{Exercise}[section]

\newtheorem{eg}{Example}[section]

% definition env
\newmdtheoremenv{defn}{Definition}

% Note env
\newmdtheoremenv{nt}{Note}

% definition env no num
\newtheorem*{defnnonum}{Definition}

% theorem envs
\newtheorem{thm}{Theorem}

% theorem envs without counter

\newtheorem{propo}[thm]{Proposition}

\newtheorem{crly}[thm]{Corollary}

\newtheorem{lemma}[thm]{Lemma}

\newtheorem{axiom}[thm]{Axiom}

\newtheorem*{thmnonum}{Theorem}

\newtheorem*{propononum}{Proposition}

\newtheorem*{crlynonum}{Corollary}

\newtheorem*{lemmanonum}{Lemma}

\newtheorem*{axiomnonum}{Axiom}


\newtheorem{note}{Note}[section]

\newtheorem{mnote}[note]{Note}

\newtheorem*{notation}{Notation}
    % warning env
\newtheorem*{warning}{Warning}

% So todo's don't get cut off
\setlength{\marginparwidth}{3cm}

% Define cuboids
\tikzset{
  annotated cuboid/.pic={
    \tikzset{%
      every edge quotes/.append style={midway, auto},
      /cuboid/.cd,
      #1
    }
    \draw [every edge/.append style={pic actions, densely dashed, opacity=.5}, pic actions]
    (0,0,0) coordinate (o) -- ++(-\cubescale*\cubex,0,0) coordinate (a) -- ++(0,-\cubescale*\cubey,0) coordinate (b) edge coordinate [pos=1] (g) ++(0,0,-\cubescale*\cubez)  -- ++(\cubescale*\cubex,0,0) coordinate (c) -- cycle
    (o) -- ++(0,0,-\cubescale*\cubez) coordinate (d) -- ++(0,-\cubescale*\cubey,0) coordinate (e) edge (g) -- (c) -- cycle
    (o) -- (a) -- ++(0,0,-\cubescale*\cubez) coordinate (f) edge (g) -- (d) -- cycle;
    \path [every edge/.append style={pic actions, |-|}]
    (b) +(0,-5pt) coordinate (b1) edge ["\cubex \cubeunits"'] (b1 -| c)
    (b) +(-5pt,0) coordinate (b2) edge ["\cubey \cubeunits"] (b2 |- a)
    (c) +(3.5pt,-3.5pt) coordinate (c2) edge ["\cubez \cubeunits"'] ([xshift=3.5pt,yshift=-3.5pt]e)
    ;
  },
  /cuboid/.search also={/tikz},
  /cuboid/.cd,
  width/.store in=\cubex,
  height/.store in=\cubey,
  depth/.store in=\cubez,
  units/.store in=\cubeunits,
  scale/.store in=\cubescale,
  width=10,
  height=10,
  depth=10,
  units=cm,
  scale=.1,
}

% highlighting shortcuts
\newcommand{\hlimpo}[1]{\textcolor{red}{\textbf{#1}}}
\newcommand{\hlwarn}[1]{\textcolor{yellow}{\textbf{#1}}}
\newcommand{\hldefn}[1]{\textcolor{blue}{\index{#1}\textbf{#1}}}
\newcommand{\hlnotea}[1]{\textcolor{green}{\textbf{#1}}}
\newcommand{\hlnoteb}[1]{\textcolor{cyan}{\textbf{#1}}}
\newcommand{\hlb}[2]{\colorbox{#1!30!background}{#2}}
\newcommand{\hlbnotea}[1]{\hlb{green}{#1}}
\newcommand{\hlbnoteb}[1]{\hlb{cyan}{#1}}
\newcommand{\hlbnotec}[1]{\hlb{yellow}{#1}}
\newcommand{\hlbnoted}[1]{\hlb{magenta}{#1}}
\newcommand{\hlbnotee}[1]{\hlb{red}{#1}}
\newcommand{\WTP}{\textcolor{bwhite}{WTP} }
\newcommand{\WTS}{\textcolor{bwhite}{WTS} }



\begin{document}

\maketitle

\tableofcontents

\renewcommand{\listtheoremname}{List of Definitions}
\listoftheorems[ignoreall,show={defn}]


\renewcommand{\listtheoremname}{\textsl{List of Theorems}}
\listoftheorems[ignoreall,
show={axiom,lemma,thm,crly,propo}
]


\chapter{Lecture 1 - Review}%
\label{chp:lecture_1_review}
% chapter lecture_1_review

\section{Sets \& tuples}%
\label{sec:sets_&_tuples}
% section sets_&_tuples



% section sets_&_tuples (end)

\begin{defn}[Tuple]\index{Tuple}\label{defn:tuple}
    A $n$ tuple is an ordered list of $n$ elements $\left( x_1, \ldots ,  x_{n}  \right) $ 
    notation 
    \begin{itemize}
        \item couple, a 2-tuple 
        \item triple, a 3-tuple 
    \end{itemize}
    \underline{Fundamental Property } 
    \[
        \left( x_1, \ldots , x_{m}  \right) = \left( y_{1} , \ldots , y_{m}  \right) \Leftrightarrow \forall i \in \left\{ 1, \ldots , m \right\}, x_{i} = y_{i} 
    \]
\end{defn}

Recall 
\[
\left\{ 1,2,3 \right\} = \left\{ 3,2,1 \right\} 
\]
But
\[
    \left( 1,2,3 \right) \neq \left( 3,2,1 \right) 
\]
In addition 
\[
    \left( 1,2,2,3 \right) \neq \left( 1,2,3 \right) 
\]
Also the comparison here doesn't even make sense since they are different sizes.

\begin{defn}[Cartesian Product]\index{Cartesian Product}\label{defn:cartesian_product}
    For sets $A,B$ 
    \[
        A\times B= \left\{ \left( a,b \right) : a \in A, b \in B \right\} 
    \]
    Note if we have $A= \emptyset $ or $B= \emptyset $ then $A\times B= \emptyset $  
\end{defn}

\begin{eg}
    \[
    A= \left\{ \pi , e \right\} \text{ and } B= \left\{ 1, \sqrt{2} , \pi  \right\} 
    \]
    \[
        A\times B = \left\{ \left( \pi , 1 \right) , \left( \pi , \sqrt{2}  \right) , \left( \pi , \pi  \right) ,\ldots  \right\} 
    \]
\end{eg}

For multiple cartesian products we have
\[
    A_{1} \times A_{2} , \ldots , A_{n} = \left\{ \left( a_1, a_2, \ldots , a_{m}  \right) : a_{i} \in A_{i}  \right\} 
\]

\begin{ex}
    Is the following true ?
    \[
        \left( A\times B \right) \times C = A\times \left( B\times C \right) = A\times B\times C
    \]
    No, observe the tuples of $\left( A\times B \right) \times C$ are of the form
    \[
        \left( \left( a,b \right) ,c \right) 
    \]
    In the same way we observe that none of them are equal . Though in a functional type of sense, they are equal as they all still convey the same fundamental idea.
\end{ex}

\section{Functions}%
\label{sec:functions}
% section functions

\begin{defn}[Function]\index{Function}\label{defn:function}
    A function is the data of two sets, $A \text{ and } B$ together with a "rule" that associates to each $x\in A$ a unique $f\left(x\right) \in B$. \\
    We define a function like this
    \[
    f : A \to B 
    \]
    Where $A$  is the domain and $B$ is the codomain.
\end{defn}

\begin{defn}[Image]\index{Image}\label{defn:image}
    The image of $E \subseteq A$ by $f$ is
    \[
    f\left(E\right) = \left\{ f\left(x\right) : x\in E \right\} 
    \]
\end{defn}

\begin{defn}[Pre-Image]\index{Pre-Image}\label{defn:pre_image}
    The pre-image of $F\in B$ by $f$ is
    \[
    f^{-1} \left(F\right) = \left\{ x\in A: f\left(x\right) \in F \right\} 
    \]
\end{defn}

\begin{defn}[Graph]\index{Graph}\label{defn:graph}
    The graph of $f$ is 
    \[
        \Gamma f = \left\{ \left( x,y \right) \in A\times B:y = f\left(x\right)  \right\} 
    \]
\end{defn}

\begin{defn}[Injective]\index{Injective}\label{defn:injective}
    A function $f : A \to B $ is injective or one-to-one 
    \[
    \forall x_1,x_2 \in A, f\left(x_1\right) = f\left(x_2\right) \implies x_1= x_2
    \]
    We have the contrapositive 
    \[
    \forall x_1,x_2 \in A, x_1 \neq x_2 \implies f\left(x_1\right) \neq f\left(x_2\right) 
    \]
\end{defn}

\begin{defn}[Onto]\index{Onto}\label{defn:onto}
    A function is surjective or onto if
    \[
    \forall y \in B, \exists x \in A, y= f\left(x\right) 
    \]
\end{defn}

\begin{defn}[Bijective]\index{Bijective}\label{defn:bijective}
    $f : A \to B $ is bijective if it is injective and surjective.
    \[
    \forall y \in B, \exists! x \in A, y= f\left(x\right) 
    \]
\end{defn}

\newpage

\begin{defn}[Inverse]\index{Inverse}\label{defn:inverse}
    $f : A \to B $ has an inverse if and only if there exists a function $g : B \to A $ such that 
    \[
        \forall x \in A, g \circ f\left( x \right) = x 
    \]
    and
    \[
    \forall x \in B, f \circ g \left(x\right) = x
    \]
    then we say that $g $ is the inverse of $f$ and $g = f^{-1} $ 
\end{defn}

% section functions (end)

\section{Function Questions}%
\label{sec:function_questions}
% section function_questions

\subsection{Visual Sets}%
\label{sub:visual_sets}
% subsection visual_sets

\begin{enumerate}
    \item not a function, observe that d maps to two different elements so it doesn't map to a unique element. 
    \item not a function, observe d is not being mapped to anything.
    \item this is a function , it is injective as we can see no element in the codomain has two arrows leading to it, it is also surjective since for every element in the codomain there is an arrow leading to it. By definition it is bijective, and it's inverse is given by turning each arrow around.
    \item It is a function, observe $f_{4} \left(c\right) = f_{4} \left(d\right) $ therefore it is not injective, though due to the same reasoning as the previous question it is surjective. Assume it's inverse exists then $g \circ f \left(c\right) = g \circ f \left(d\right) $   but then $c = d$ so a contradiction. 
    \item It is a function, it is injective, though not surjective nor bijective, the inverse does not exist.
    \item It is a function, not injective nor surjective therefore not bijective and the inverse must not exist.
\end{enumerate}

% subsection visual_sets (end)

\subsection{Defined sets}%
\label{sub:defined_sets}
% subsection defined_sets

\begin{enumerate}
    \item $f_{7} $ I believe this is a function, if we take an element from the codomain for example $a e^{b} $ this must only have come from $\left( a,b \right) $.  It is not surjective 
        \[
        f\left(e,0\right) = f\left(1,1\right) 
        \]
        It is surjective, let $k\in \mathbb{R} $ then we have $f\left(k,0\right) $ 
    \item $f_{8} $ I believe this is a function, take $f\left(j,k\right) $ this maps to the unique element $(e^{j}, k^2) $. Not injective consider $f\left(0, 1\right) \text{ and } f\left(0, -1\right) $. Not surjective, observe $e^{x} > 0$ therefore nothing maps to $\left( -1, p \right) $ 
% section function_questions (end)
\end{enumerate}

% subsection defined_sets (end)

\subsection{Image and Inverse Image}%
\label{sub:image_and_inverse_image}
% subsection image_and_inverse_image

\begin{enumerate}
    \item $f\left(\left\{ a,c,d \right\} \right) $ by definition
        \[
        \left\{ f\left(x\right) : x\in \left\{ a,b,c \right\}  \right\} = \left\{ 1,2 \right\} 
        \]
    \item $f^{-1} \left(\left\{ 2,3,4 \right\} \right) $ by definition 
        \[
        \left\{ x\in \left\{ a,b,c,d \right\} :f\left(x\right) \in \left\{ 2,3,4 \right\}  \right\} = \left\{ c,d,b \right\} 
        \]
    \item By definition we have 
        \[
            \left\{ \left( 1,3 \right) , \left( 2,5 \right) , \left( 3,1 \right) , \left( 4,5 \right)  \right\} 
        \]
    \item This first is a graph, by inspection There is a unique element in the codomain for every element in the domain.
    \item The second is not, we observe $f(2) = j \text{ and } f\left(2\right) = k $ but $j\neq k$  
\end{enumerate}


% subsection image_and_inverse_image (end)
\subsection{Final Four}%
\label{sub:final_four}
% subsection final_four

\begin{enumerate}
    \item Not injective $f\left(a,b,z\right) = f\left(a,b,x\right) ,x\neq z $. It is surjective, it is not bijective so the inverse does not exist .
    \item Initially, I thought it may be possible it it is not injective since we have $x^2 $ though the $e^{x} $ showed me that idea would not work. So I'll prove it's injective. Assume 
        \[
            f\left(a,b\right) = f\left( j,k \right) \Leftrightarrow \left( e^{a} , \left( a^2  + 1 \right) b \right) = \left( e^{j} ,\left( j^2  + 1 \right) k \right) 
        \]
        we have $e^{a} = e^{j} \therefore a = j $, though we also have
        \[
            \left( a^2  + 1 \right) b = \left( j^2  + 1 \right) k \Leftrightarrow \left( a^2  + 1 \right) b= \left( a^2  + 1 \right) k \Leftrightarrow b= k
        \]
        therefore it's injective. We know $e^{x} > 0$ therefore $\left( a,k \right), a \le 0$ is not mapped to. It's not bijective so the inverse does not exist.
    \item I'll try the same idea as the last question, 
        \[
            \left( a + b,  - a \right) = \left( j + k,  - j \right) 
        \]
        therefore $a = j$ then we have $a + b= j + k \Leftrightarrow b= k$ so its injective, we will show it's surjective, let $\left( l,m \right) \in \mathbb{R} ^2 $ then take $x=  - m$ and $y= m + l$ so $f\left(x,y\right) = \left(  - m  + m + l,  - \left(  - m \right)  \right) = \left( l,m \right) $, find the inverse next.
\end{enumerate}

% subsection final_four (end)

\section{Geometry in Higher Dimensions}%
\label{sec:geometry_in_higher_dimensions}
% section geometry_in_higher_dimensions

\begin{defn}[$R^{n} $ ]\index{$R^{n} $ }\label{defn:_r_n_}
    \[
        R^{n} = \left\{ \left( x_{1},  x_{2},  \dotsc   x_{n - 1},  x_{n},  \right): x_{i} \in \mathbb{R}   \right\} 
    \]
    Also note that it could be thought of as, though we get nesting couples.
    \[
        \underbracket{ \mathbb{R} \times \mathbb{R} \times \ldots \times \mathbb{R} \times \mathbb{R} }_{n \text{ times } }
    \]
\end{defn}

\begin{itemize}
    \item $\mathbb{R} ^{2} $ 
        \begin{itemize}
            \item We think of this as a plane, perhaps the $x, y$ coordinate system
            \item $\left( x,y \right) \in \mathbb{R} ^2 $ 
        \end{itemize}
    \item $\mathbb{R} ^{3} $ 
        \begin{itemize}
            \item We can think of this as 3d, space the space we live in
            \item $\left( x,y,z \right) \in  \mathbb{R} ^{3}  $ 
        \end{itemize}
    \item $\mathbb{R} ^{n} $ 
        \begin{itemize}
            \item This is $n$-dimensional space, hard to visualize, though it makes sense in an algebraic sense.
            \item $\left( x_{1} , x_{2} , \dotsc  , x_{n - 1} , x_{n}  \right) \in \mathbb{R} ^{n} $ 
            \item We denote an $n$-tuple of $\mathbb{R} ^{n} $ like this $\vec{x} $  
        \end{itemize}
    \item $\vec{e_{n} } = \left( 0, \ldots , 0, 1, \ldots  \right) $ where $1$ is at the n-th entry.
\end{itemize}

\subsection{Operations on n-tuples}%
\label{sub:operations_on_n_tuples}
% subsection operations_on_n_tuples

Let $a = \left( a_{1} , a_{2} , \dotsc  , a_{n - 1} , a_{n}  \right) $, $b = \left( b_{1} , b_{2} , \dotsc  , b_{n - 1} , b_{n}  \right) $ and $\lambda \in \mathbb{R} $ 

\begin{itemize}
    \item Addition:
       \[
           a + b = \left( a_{1} + b_{1} , a_{2} + b_{2} , \dotsc  , a_{n - 1} + b_{n - 1} , a_{n} + b_{n}  \right) 
       \]
    \item scalar multiplication 
        \[
            \lambda a = \left( \lambda a_{1} , \lambda a_{2} , \dotsc  , \lambda a_{n - 1} , \lambda a_{n}  \right) 
        \]
\end{itemize}

\begin{defn}[Dot Product]\index{Dot Product}\label{defn:dot_product}
    \[
    a \cdot b = \underbracket{\sum_{i=0}^{n} a_{i} b_{i}}_{\chi}  = b \cdot a 
    \]
    Note $\chi \in \mathbb{R} $ 
\end{defn}

\begin{itemize}
    \item $\left( \lambda a \right)  \cdot b= \lambda \left( a  \cdot b \right) $ 
    \item $\left( a + b \right)  \cdot c= a \cdot c + b \cdot c$ 
    \item $a\neq \vec{0} \implies a \cdot a > 0$ also $a \cdot a = 0 \implies a = \vec{0} $ additionally $\vec{0}  \cdot a = 0$ 
\end{itemize}

\begin{defn}[Norm]\index{Norm}\label{defn:norm}
    Let $a \in \mathbb{R}^{n}  $ 
    \[
    \left\Vert a \right\Vert = \sqrt{a \cdot a} = \sqrt{a_{1}^2   +  a_{2}^2   +  \dotsb   +  a_{n - 1}^2   +  a_{n}^2  } 
    \]
    geometrically this is the length of $a$, also $\left\Vert a - b \right\Vert $ is the distance between $a \text{ and } $ 
\end{defn}


% subsection operations_on_n_tuples (end)

\subsection{Important properties of the norm}%
\label{sub:important_properties_of_the_norm}
% subsection important_properties_of_the_norm

Prove the following.
\begin{enumerate}
    \item $\left\Vert a \right\Vert \ge 0$ 
    \item $\left\Vert a \right\Vert = 0 \implies a = \vec{0} $ 
    \item $\left\Vert \lambda a \right\Vert = \left| \lambda  \right| \left\Vert a \right\Vert  $ 
    \item $\left\Vert a + b \right\Vert \le \left\Vert a \right\Vert  + \left\Vert b \right\Vert $, (hint use 5)
    \item $\left| a \cdot b \right| \le \left\Vert a \right\Vert \left\Vert b \right\Vert $ (Cauchy-Schwarts inequality)
    \item $a \cdot e_{j} = a_{j} $ 
    \item $e_{j}  \cdot e_{j} = 1$ 
    \item for $i\neq j$,  $e_{j}  \cdot e_{i} = 0$ 
    \item $a \cdot b = \frac{1}{4}\left( \left\Vert a + b \right\Vert ^2  - \left\Vert a - b \right\Vert ^2  \right) $ (Polarization identity )
\end{enumerate}

% subsection important_properties_of_the_norm (end)

% section geometry_in_higher_dimensions (end)

% chapter lecture_1_review (end)


\end{document}
