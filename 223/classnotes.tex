% !TEX TS-program = pdflatex
\documentclass[notoc,notitlepage]{tufte-book}
% \nonstopmode % uncomment to enable nonstopmode

\usepackage{classnotetitle}

\newcommand\mat[1]{\begin{bmatrix}#1\end{bmatrix}}

\title{MAT223 - Linear Algebra}
\author{Callum Cassidy-Nolan}
\subtitle{Classnotes for Summer 2019}
\credentials{Computer Science}
\institution{University of Toronto}

\input{latex-classnotes-preamble.tex}

\begin{document}
\input{latex-classnotes-header.tex}

\chapter{Week 9}%
\label{chp:week_9}
% chapter week_9

Linear transformations and such

\section{Linear Transformations}%
\label{sec:linear_transformations}
% section linear_transformations

\begin{defn}[Linear Transformation]\index{Linear Transformation}\label{defn:linear_transformation}
    Let $V$ and $W$ be subspaces. A function $\mathcal{T} : V \to W $ is called
    a linear transformation if for all $\vec{u}, \vec{v} \in V$ and $a \in \mathbb{R}$ it satisfies
    \begin{enumerate}
        \item $\mathcal{T}(\vec{u} + \vec{v}) = \mathcal{T}(\vec{u}) + \mathcal{T}(\vec{v})$ 
        \item $\mathcal{T}(a \vec{u}) = a \mathcal{T}(\vec{u})$ 
    \end{enumerate}
\end{defn}

\begin{eg}
    We'll show that $\mathcal{T} : \mathbb{R}^2 \to \mathbb{R}^2 $ where
    $\mathcal{T} \mat{ x \\ y } = \mat{ x + 2 \\ y }$ is not a linear
    transformation.

    Let $\vec{j} = \mat{ 0 \\ 0 }, \vec{k} = \mat{ 0 \\ 0 }$ we have that
    \begin{equation*}
        \mathcal{T}(\mat{ 0 \\ 0 } + \mat{ 0 \\ 0 }) = \mat{ 2 \\ 0 }       
    \end{equation*}
    But then we can see that 
    \begin{equation*}
        \mathcal{T}(\mat{ 0 \\ 0 }) + \mathcal{T}(\mat{ 0 \\ 0 }) = \mat{ 2 \\ 0
        } + \mat{ 2 \\ 0 } = \mat{ 4 \\ 0 }         
    \end{equation*}
    Then we conclude that $\mathcal{T}(\vec{j} + \vec{k}) \neq
    \mathcal{T}(\vec{j}) + \mathcal{T}(\vec{k})$ 
\end{eg}

\begin{eg}
    We'll show that $\mathcal{P}$ is a linear transformation 
    \sidenote{We'll show that it is closed under addition and multiplication}
    \begin{equation*}
        \mathcal{P}(\mat{ x \\ y }) = \mathit{comp}_{\vec{u}} {\mat{ x \\ y }} 
    \end{equation*}
    Let $\vec{j}, \vec{k} \in \mathbb{R}^2$ we know that 
    \begin{equation*}
        \mathit{comp}_{\vec{u}} {\vec{j}}  = \left( \frac{\vec{u} \cdot
        \vec{j}}{\left\Vert \vec{u} \right\Vert^2} \right) \vec{u} \text{ and }
        \mathit{comp}_{\vec{u}} {\vec{k}}  = \left( \frac{\vec{u} \cdot
        \vec{k}}{\left\Vert \vec{u} \right\Vert^2} \right) \vec{u}
    \end{equation*}
    And thus their product yields
    \begin{align*}
        \mathit{comp}_{\vec{u}} {\vec{j}}  + \mathit{comp}_{\vec{u}} {\vec{k}}
        &= \left( \frac{\vec{u} \cdot \left( \vec{j} + \vec{k}
        \right)}{\left\Vert \vec{u} \right\Vert^2} \right) \vec{u}\\
    \end{align*}
    Which is equal to 
    \begin{equation*}
        \mathit{comp}_{\vec{u}} {\left( \vec{j} + \vec{k} \right)}     
    \end{equation*}
    We must then show that it holds under multiplication let $\alpha \in
    \mathbb{R}$ and we know that 
    \begin{equation*}
        \alpha \mathit{comp}_{\vec{u}} {\vec{j}} = \alpha \left( \frac{\vec{u}
        \cdot \vec{j}}{ \left\Vert \vec{u} \right\Vert^2} \right) \vec{u} =
        \left( \frac{\vec{u} \cdot \alpha \vec{j}}{\left\Vert \vec{u}
        \right\Vert^2} \right) \vec{u} = \mathit{comp}_{\vec{u}} {\alpha \vec{j}} 
    \end{equation*}
\end{eg}

% section linear_transformations (end)

% chapter week_9 (end)

\appendix

\backmatter

\fancyhead[LE]{\thepage \enspace \textsl{\leftmark}}

\nocite{*}

\bibliography{references}

\printindex

\end{document}
% vim:tw=80:fdm=syntax
